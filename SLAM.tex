%\documentclass[12pt,draft]{article}
\documentclass[12pt]{article}
\usepackage{CJK}
\AtBeginDvi{\input{zhwinfonts}}
\usepackage{mathrsfs}
\usepackage{amsmath,amsthm,amsfonts,amssymb}
\usepackage{geometry}
\usepackage{fancyhdr}
\usepackage{indentfirst}
\usepackage{float}
\usepackage[dvips]{graphicx}
\usepackage{subfigure}
\usepackage[font=small]{caption}
\usepackage{threeparttable}
\usepackage{cases}
\usepackage{multicol}
\usepackage{url}
\usepackage{amsmath}
\usepackage{bm}
\usepackage{xcolor}
\usepackage{overpic}
\usepackage[utf8]{inputenc}
\usepackage[american]{babel}
%\usepackage[round]{natbib}
\usepackage{amsmath}
\DeclareMathOperator*{\argmax}{argmax} % thin space, limits underneath in displays
\DeclareMathOperator*{\argmin}{argmin} % thin space, limits underneath in displays
\usepackage{graphicx}
\numberwithin{equation}{section}
\geometry{left=1.5cm,right=1.5cm,top=1.5cm,bottom=1.5cm}
\setlength{\parskip}{0.3\baselineskip}
\setlength{\headheight}{15pt}
\usepackage{color}   %May be necessary if you want to color links
\usepackage{hyperref}
\hypersetup{
    colorlinks=true, %set true if you want colored links
    linktoc=all,     %set to all if you want both sections and subsections linked
    linkcolor=blue,  %choose some color if you want links to stand out
}
\newtheorem{Proposition}{Proposition}
\setlength{\parindent}{10ex}
\begin{document}\small
\title{SLAM Notes}
\author{rockking.jy@gmail.com}
\pagestyle{fancy}\fancyhf{}
\lhead{}\rhead{rockking.jy@gmail.com}
\lfoot{\textit{}}\cfoot{}\rfoot{\thepage}
\renewcommand{\headrulewidth}{1.pt}
\renewcommand{\footrulewidth}{1.pt}
\maketitle
\tableofcontents
%=======================================
\section{Probabilistic Data Association for Semantic SLAM \cite{bowman2017probabilistic}}
\subsection{Notations}
$\mathcal{Z}=\{z_k\}^K_{k=1}$: K Sensor Measurements. \par
$\mathcal{X}=\{\bm{x_t}\}^T_{t=1}$: T sensor poses. \par
$\mathcal{L}=\{l_m\}^M_{m=1}$: M Static landmarks, consist of its position $l^p \in \mathbb{R}^3$ and a class label $l^c$ from a discrete set $\mathbb{C}=\{1, \cdots , C\}$. \par
$\mathcal{D}=\{(\alpha_k, \beta_k)\}^K_{k=1}$: K Data associations, the measurement $z_k$ of landmark $l_{\beta_k}$ was obtained from sensor state $x_{\alpha_k}$. \par
SLAM Task Definition: Given $\mathcal{Z}$, estimate $\mathcal{X}$, $\mathcal{L}$ and $\mathcal{D}$:
\begin{equation}
	\hat{\mathcal{X}}, \hat{\mathcal{L}}, \hat{\mathcal{D}} = \\
		\argmax_{\mathcal{X},\mathcal{L},\mathcal{D}} log\ p(\mathcal{Z}|\mathcal{X},\mathcal{L},\mathcal{D})
\end{equation} \par
Common approach to this task:
\begin{subequations}
\begin{align}
	\hat{\mathcal{D}} &= 
		\argmax_{\mathcal{D}} log\ p(\mathcal{D}|\mathcal{X}^0,\mathcal{L}^0,\mathcal{Z}) \\
	\hat{\mathcal{X}}, \hat{\mathcal{L}} &= 
		\argmax_{\mathcal{X},\mathcal{L}} log\ p(\mathcal{Z}|\mathcal{X},\mathcal{L},\hat{\mathcal{D}})
\end{align}
\end{subequations} \par
In order to revisit association decisions to avoid detrimental effect from incorrectly chosen data association, instead of simple one step process above, it is possible to perform \textbf{coordinate descent}:
\begin{subequations}
\begin{align}
	\mathcal{D}^{i+1} &= \argmax_{\mathcal{D}} log\ p(\mathcal{D}|\mathcal{X}^i,\mathcal{L}^i,\mathcal{Z}) \\
	\mathcal{X}^{i+1}, \mathcal{L}^{i+1} &= \argmax_{\mathcal{X},\mathcal{L}} log\ p(\mathcal{Z}|\mathcal{X},\mathcal{L},\mathcal{D}^{i+1})
\end{align}
\end{subequations} \par
To resolve the problem of ambiguous measurements (if the measurements are ambiguous, they will be discarded), utilize the whole density of $\mathcal{D}$ visa \textbf{EM}(expectation maximization) to averages over all possible data associations $\mathbb{D}$:
\begin{align}\begin{split}\label{eq:em1}
	\mathcal{X}^{i+1}, \mathcal{L}^{i+1} &= 
	\argmax_{\mathcal{X},\mathcal{L}} \mathbb{E}_{\mathcal{D}} [log\ p(\mathcal{Z}|\mathcal{X},\mathcal{L},\mathcal{D})|\mathcal{X}^i,\mathcal{L}^i,\mathcal{Z}] \\
	&= \argmax_{\mathcal{X},\mathcal{L}} \sum_{\mathcal{D} \in \mathbb{D}} \ p(\mathcal{D}|\mathcal{X}^i,\mathcal{L}^i,\mathcal{Z}) log\ p(\mathcal{Z}|\mathcal{X},\mathcal{L},\mathcal{D})
\end{split}\end{align}
using $p(\mathcal{Z}|\mathcal{X},\mathcal{L},\mathcal{D})=\prod_kp(z_k|x_{\alpha_k}, l_{\beta_k})$, we have:
\begin{equation}\label{eq:em2}
	\argmax_{\mathcal{X},\mathcal{L}} \sum_{\mathcal{D} \in \mathbb{D}} \sum_{k=1}^K \ p(\mathcal{D}|\mathcal{X}^i,\mathcal{L}^i,\mathcal{Z})log\ p(z_k|x_{\alpha_k}, l_{\beta_k})  \\
	= \argmax_{\mathcal{X},\mathcal{L}} \sum_{k=1}^K \sum_{j=1}^{M} \ w^i_{kj} log\ p(z_k|x_{\alpha_k}, l_j) 
\end{equation}
where, $w^i_{kj}=\sum_{\mathcal{D} \in \mathbb{D}(k,j)} p(\mathcal{D}|\mathcal{X}^i,\mathcal{L}^i,\mathcal{Z})$, and $\mathbb{D}(k,j)=\{\mathcal{D} \in \mathbb{D} | \beta_k = j\} \subseteq \mathbb{D}$ is the set of all data associations such that measurement $k$ is assigned to landmark $j$. \par
The EM formulation (\ref{eq:em1}) allows us to solve \textbf{iteratively} from the proposition:
\begin{Proposition}
	If $p(\mathcal{D}|\mathcal{X}^i,\mathcal{L}^i)$ is uniform, the maximizers of the EM formulation (\ref{eq:em1}) is euqal to the optimization:
	\begin{equation}
		\mathcal{X}^{i+1}, \mathcal{L}^{i+1} = \argmax_{\mathcal{X},\mathcal{L}} per(Q^i(_{\mathcal{X},\mathcal{L}}))
	\end{equation}
\end{Proposition} \par
So EM algorithm has the following steps:
\begin{enumerate}
	\item E step: estimate the data association distribution $p(\mathcal{D}|\mathcal{X}^i,\mathcal{L}^i,\mathcal{Z})$ in the form of the $w^i_{kj}$.
	\item M step: maximize the expected measurement log likelihood (\ref{eq:em2}).
\end{enumerate}
%-----------------------------------------------------------
\subsection{Semantic SLAM}
$\mathcal{I}=\{\mathcal{I}_t\}^T_{t=1}$: inertial information from IMU. \par
$\mathcal{Y}=\{\mathcal{Y}_t\}^T_{t=1}$: geometric information, extract ORB features and match them. \par
$\mathcal{S}=\{\mathcal{S}_t\}^T_{t=1}$: semantic information  from object recognition algorithm, $\bm{s_k}=(s^c_k, s^s_k, s^b_k) \in \mathcal{S}_t$ represents $(class, score, bbox)$ \par
If the data association $\mathcal{D}_k=(\alpha_k, \beta_k)$ of measurement $\bm{s_k}$ is known, the measurement likelihood can be decomposed as follows:
\begin{equation}
	p(\bm{s_k}|x_{\alpha_k}, l_{\beta_k})=p(s^c_k|l^c_{\beta_k}) p(s^s_k|l^c_{\beta_k},s^c_k) p(s^b_k|x_{\alpha_k},l^c_{\beta_k})
\end{equation}
where, $p(s^c_k|l^c_{\beta_k})$ is the confusion matrix of the object detector, $p(s^s_k|l^c_{\beta_k},s^c_k)$ is score distribution, both of them are learned offline. And $p(s^b_k|x_{\alpha_k},l^c_{\beta_k})$ is assumed as normally distribution. \par
$\mathcal{I}$ an $\mathcal{Y}$ used to track the sensor trajectory \textbf{locally}, while $\mathcal{S}$ used to construct a map of objects to perform loop closure that is robust to ambiguities and viewpoint efficiently. \par
%-----------------------------------------------------------
\subsection{Semantic SLAM using EM}
\subsubsection{Proposition}
\begin{Proposition}
	If
	\begin{itemize}
		\item $p(\mathcal{D}|\mathcal{X},\mathcal{L})$ is uniform.
		\item Semantic measurement data associations are independent across keyframes: $p(\mathcal{D}|\mathcal{S},\mathcal{X},\mathcal{L})=\prod^T_{t=1}(\mathcal{D}_t|\mathcal{S}_t,\mathcal{X},\mathcal{L})$.
	\end{itemize}
	the semantic SLAM problem can be solved via the EM algorithm by iteratively solving for:
	\begin{enumerate}
		\item E-step: Data association weights $w^t_{kj}$:
		\begin{equation}
			w^{t,(i)}_{kj}=\sum_{l^c \in \mathbb{C}} \sum_{\mathcal{D}_t \in \mathbb{D}_t(k,j)} k^{(i)}(\mathcal{D}_t, l^c) \ \ \ \ \forall t, k, j
		\end{equation}
		where,
		\begin{equation}
			k^{(i)}(\mathcal{D}_t,l^c)=\frac{p(S_t|\mathcal{X}^{(i)},\mathcal{L}^{(i)}, \mathcal{D}_t)}{\sum_{l^c \in \mathbb{C}} \sum_{\mathcal{D}_t \in \mathbb{D}_t} p(S_t|\mathcal{X}^{(i)},\mathcal{L}^{(i)}, \mathcal{D}_t)}
		\end{equation}
		$\mathbb{D}_t$: the set of all possible data associations for measurements received at timestep t. \par
		$\mathbb{D}_t(k,j) \subseteq \mathbb{D}_t$: the set of data associations that measurement i is assigned to landmark j. 
		\item M-step: Sensor states $\mathcal{X}$ and landmark positions $l^p_{1:M}$:
		\begin{equation} \label{eq:mstep}
			\mathcal{X}^{(i+1)}, {l}^{p,(i+1)}_{1:M}=\argmin_{\mathcal{X}, l^p_{1:M}} \sum^T_{t=1} \sum_{\bm{s_k} \in S_t} \sum^M_{j=1} -w^{t,(i)}_{kj} log \ p(\bm{s_k}|\bm{x_t}, l_j) - log\ p(\mathcal{Y}|\mathcal{X}) - log\ p(\mathcal{I}|\mathcal{X})
		\end{equation}
	\end{enumerate}	
\end{Proposition} \par
\subsubsection{Analysis of the proposition}
Pose graph:  \par
A set of vertices $\mathcal{V}$ corresponds to the set of optimization variables. \par
A set of factors $\mathcal{F}$ is a generalization of the edges that allow connectivity between more than two vertices, corresponds to the set of individual components of the cost function. \par
A factor f depends on a subset of $\mathcal{V}$ such that the entire optimization is of the form:
\begin{equation}
	\hat{\mathcal{V}}=\argmin_{\mathcal{V}} \sum_{f \in \mathcal{F}} f(\mathcal{V})
\end{equation}
\begin{enumerate}
	\item E step
	\item M step
		\begin{itemize}
		\item Semantic Factors \par
			A measurement $\bm{s_k}$ from sensor state $\bm{x_i}$ defines factors 
			$f^s_{kj}(\bm{x_i}, l_j)$ for each visible landmark j. \par
			Let $h_{\pi}(\bm{x}, l^p)$ be the projection of a landmark $l^p$ onto a camera at pose 
			$\bm{x}$. \par
			Assume the camera measurement of a landmark $l^p$ from camera pose $\bm{x}$ is 
			Gaussian distributed with mean $h_{\pi}(\bm{x}, l^p)$ and covariance $\bm{R_s}$. \par
			Thus, a camera factor corresponding to sensor state t, measurement k, and landmark j, 
			becomes:
			\begin{equation}
				f^s_{kj}(\mathcal{X}, \mathcal{L})= -w^{t,(i)}_{kj} log \ p(s_k^b|\bm{x_t}, l_j^p)
				=||s_k^b-h_\pi (\bm{x_t}, l_j)||^2_{\bm{R_s}/w^{t,(i)}_{kj}}
			\end{equation}
		\item Geometric Factors \par
			The term corresponding to geometric factors in (\ref{eq:mstep}) rewrite as:
			\begin{align}\begin{split}
			- log\ p(\mathcal{Y}|\mathcal{X}) 
			&=-\sum^{N_y}_{i=1}\sum_{k:\beta^y_k=i} log\ p(y_k|x_{\alpha^y_k}) \\
			&=\sum^{N_y}_{i=1}f^y_i(\mathcal{X}) \\
			&=\sum^{N_y}_{i=1}\sum_{k:\beta^y_k=i}||\bm{y}_k-h_\pi (\bm{x}_{\alpha_k^y}, \rho_i)||^2_{\bm{R}_y}
			\end{split}\end{align}
			where $N_y$ is the total number of distinct feature trackers. $\rho_{\beta^y_k}$ be the 3D position in the frame of landmark that generated measurement $\bm{y}_k$, and assuming the projection has Gaussian pixel noise with covariance $\bm{R}_y$. \par
		\item Inertial Factors \par
			\begin{equation}
				f^{\mathcal{I}}_i(\mathcal{X})
				=-log \ p(\mathcal{I}_{ij}|\mathcal{X})
				=||\bm{r}_{\mathcal{I}_{ij}}||^2_{\sum_{ij}}
			\end{equation}
			where $\bm{r}_{\mathcal{I}_{ij}}$ is the inertial residuals on the rotation($\bm{r}_{\Delta R_{ij}}$), velocity($\bm{r}_{\Delta v_{ij}}$) and position($\bm{r}_{\Delta p_{ij}}$) differences between two keyframes as a function of the pose $\bm{x_i}$ and $\bm{x_j}$.
		\end{itemize}
\end{enumerate}
The full pose graph optimization is then becomes: 
	\begin{equation}
			\hat{\mathcal{X}}_{1:T}, \hat{l}_{1:M}=\argmin_{\mathcal{X}, l_{1:M}} \sum^K_{k=1} \sum^M_{j=1}f^s_{kj}(\mathcal{X}, l^p_{1:M}) + \sum^{N_y}_{i=1} f^y_i(\mathcal{X}) + \sum^{T-1}_{t=1} f^{\mathcal{I}}_t(\mathcal{X})
	\end{equation}
%=======================================
\renewcommand\refname{Reference}
\bibliographystyle{ieeetr}
\bibliography{SLAM} 
\clearpage
\end{document}