%\documentclass[12pt,draft]{article}
\documentclass[12pt]{article}
\usepackage{CJK}
\AtBeginDvi{\input{zhwinfonts}}
\usepackage{mathrsfs}
\usepackage{amsmath,amsthm,amsfonts,amssymb}
\usepackage{geometry}
\usepackage{fancyhdr}
\usepackage{indentfirst}
\usepackage{float}
\usepackage{listings}
\usepackage[dvips]{graphicx}
\usepackage{subfigure}
\usepackage[font=small]{caption}
\usepackage{threeparttable}
\usepackage{cases}
\usepackage{multicol}
\usepackage{url}
\usepackage{amsmath}
\usepackage{commath}
\usepackage{bm}
\usepackage{xcolor}
\usepackage{overpic}
\usepackage[utf8]{inputenc}
\usepackage[american]{babel}
\usepackage[round]{natbib}
\usepackage{graphicx}
\numberwithin{equation}{section}
\geometry{left=1.5cm,right=1.5cm,top=1.5cm,bottom=1.5cm}
\setlength{\parskip}{0.3\baselineskip}
\setlength{\headheight}{15pt}
\usepackage{color}   %May be necessary if you want to color links
\usepackage{hyperref}
\hypersetup{
    colorlinks=true, %set true if you want colored links
    linktoc=all,     %set to all if you want both sections and subsections linked
    linkcolor=blue,  %choose some color if you want links to stand out
}
\setlength{\parindent}{4ex}
\begin{document}\small
\title{Visual Tracking Notes}
\author{Yan JIN}
\pagestyle{fancy}\fancyhf{}
\lhead{}\rhead{JIN Yan}
\lfoot{\textit{}}\cfoot{}\rfoot{\thepage}
\renewcommand{\headrulewidth}{1.pt}
\renewcommand{\footrulewidth}{1.pt}
\maketitle
\tableofcontents
%=======================================
\section{HOG feature \citep{dalal2005histograms} \citep{felzenszwalb2010object}} 
\subsection{Original HOG feature  \citep{dalal2005histograms} } 
Basic idea: local object appearance and shape can often be characterized rather well
by the distribution of local intensity gradient or edge directions, even without precise knowledge
of the corresponding gradient or edge position. \par
Input - detection window with size: $64 \times 128$. 
\begin{enumerate}
	\item Divide the detection window into $16 \times 16$ blocks of $50\%$ overlap.
		\begin{itemize}
			\item $7 \times 15 = 105$ blocks in total ($(64 \div (16 \div 2) - 1) \times 
				(128 \div (16 \div 2) - 1)$. 
		\end{itemize}
	\item Each block consists of $2 \times 2$ cells with size 8 pixels $ \times $ 8 pixels.
	\item Compute gradients for each pixel.
		\begin{itemize}
			\item Using $[-1, 0, 1]$ gradient filter.
		\end{itemize}
	\item Quantize the gradient orientation into 9 orientations bins in $0^\circ-180^\circ$
	(with $20^\circ$ interval).
		\begin{itemize}
			\item The vote is the gradient magnitude. \\
			For color images, calculate separate gradients for each color channel, and take the one with 
			the largest norm at the pixel's gradient vector.
			\item Interpolate votes bi-linearly between neighboring bin center. \\
			 For example, suppose a pixel have orientation $80^\circ$, next to it has $70^\circ$ and 
			 $90^\circ$ bins, then: $85-70=15$, $90-85=5$, so, for bin $70^\circ$ it contributes:
			  $15/(15+5)=3/4 \times$ gradient magnitude, and for bin $90^\circ$: $5/(15+5)=1/4 \times$
			  gradient magnitude.
			\item The vote can also be weighted by Gaussian to down-weight near the edge.
		\end{itemize}
	\item For each cell accumulating the 1-D histogram of gradient directions or edge orientations over
	the pixels of the cell.
	\item Normalization for blocks, for better invariance to illumination, shadowing etc..
		\begin{itemize}
			\item L2-Hys (Lowe-style clipped L2 norm) block normalization. That is, L2-norm followed by 
			clipping (limiting the maximum values of v to 0.2) and renormalizing.
		\end{itemize}		
	\item Concatenate histograms.
		\begin{itemize}
			\item Feature dimension: $36(=4 \times 9)$ for each block (in this example, 105 blocks in total,
			so $36 \times 105$ features in total for this input detection window).
		\end{itemize}	
\end{enumerate}
%--------------------------------
\subsection{Improved HOG feature \citep{felzenszwalb2010object}}
Basic idea: using 13-dimensional feature instead of previous 36-dimensional with no significant effect, and
add  contrast sensitive and contrast insensitive features to a 31-dimensional feature vector to improve the 
performance. \par
Input - image size: $w \times h$.
\begin{enumerate}
	\item Pixel-Level Feature Maps. 
		\begin{itemize}
			\item For each pixel (x, y), let $\theta(x,y)$ be the orientation and $r(x,y)$ be the magnitude of 
			the intensity gradient using $[-1, 0, 1]$ as gradient filter.
			\item The gradient calculated at the pixel is discretized into one of p values by contrast sensitive
			($B_1$) or contrast insensitive ($B_2$):
				\begin{equation}
					B_1(x,y) = round(\frac{p \theta(x,y)}{2 \pi})\mod p 
				\end{equation}
				\begin{equation}
					B_2(x,y) = round(\frac{p \theta(x,y)}{\pi})\mod p
				\end{equation}
			\item The feature vector $F(x, y)_b$ at pixel (x, y) is: 
				\begin{equation}
					F(x,y)_b=
					\begin{cases}
						r(x,y) & \text{if } b = B(x,y) \\
						0 & \text{otherwise}
    					\end{cases}
				\end{equation}
			where $b \in \{0, \cdots , p-1 \}$.
			\item Now we have a pixel-level feature map for the image.
		\end{itemize}
	\item Spatial Aggregation.
		\begin{itemize}
			\item Let $k > 0$ be the cell size.
			\item Aggregate pixel-level feature vectors to obtain a cell-based feature map C(i,j).
			\item In C(i,j), where $0 \le i \le \lfloor{(w-1)/k}\rfloor$, $0 \le j \le \lfloor{(h-1)/k}\rfloor$.
			\item Pixel (x, y) maps to $(\lfloor{x/k}\rfloor, \lfloor{y/k}\rfloor)$.
			\item The feature vector at a cell is the sum (or average) of the pixel-level features in that cell.
			\item Each pixel contributes to the feature vectors in the four cells around it using 
			bilinear interpolation.
			\item This aggregation provides some invariance to small deformations and reduces the size
			of a feature map.
		\end{itemize}
	\item Normalization and Truncation.
		\begin{itemize}
			\item Normalization provides the invariance to gain, while gradients provides invariant to bias.
			\item Let $T_{\alpha}(v)$ denote the truncation of a vector  v by $\alpha$.
			\item The feature map is obtained by concatenating the result of normalizing the map C with respect
			 to each normalization factor followed by truncation:
			 	\begin{equation}
					H(i,j)=
					\begin{bmatrix}
						T_{\alpha}(C(i,j)/N_{-1, -1}(i,j)) \\
						T_{\alpha}(C(i,j)/N_{+1, -1}(i,j)) \\
						T_{\alpha}(C(i,j)/N_{+1, +1}(i,j)) \\
						T_{\alpha}(C(i,j)/N_{-1, +1}(i,j)) \\
					\end{bmatrix}
				\end{equation}
			\item We use four normalization factors for C(i,j): $N_{\delta, \gamma}(i,j)$ with 
			$\delta, \gamma \in \{-1, 1\}$,
				\begin{equation}
					N_{\delta, \gamma}(i,j)=(\norm{C(i,j)}^2+\norm{C(i+\delta,j)}^2+
									\norm{C(i,j+\gamma)}^2+\norm{C(i+\delta,j+\gamma)}^2)^{1/2}
				\end{equation}
			Each factor measures the "gradient energy" in a square block of four cells containing (i,j).
			\item Commonly, HOG features using $p=9$, discretized with $B_2$, cell size $k=8$ and truncation 
			$\alpha = 0.2$, so we have 36-dimensional feature vector (9 orientations $ \times 4$ 
			normalizaions).
		\end{itemize}		
	\item PCA and Analytic Dimensionality Reduction.
		\begin{itemize}
			\item 9 contrast insensitive orientations, 18 contrast sensitive orientations, 4 normalization factors, 
			so we have $4 \times (9+18)=108$ dimensional feature vectors. 
			\item Use an analytic projection of 108 dimensional features to:
				\begin{enumerate}
			 		\item 27 (=9+18) sums over 4 different normalizations.
					\item 4 dimensional features sums over 9 contrast insensitive orientations. 
				\end{enumerate}
			\item In total, the final feature map has 31-dimensional vectors (27+9).
		\end{itemize}	
\end{enumerate}
%=======================================
\section{CN feature \citep{van2007learning}}
%=======================================
\section{C-COT \citep{DanelljanECCV2016}}

\begin{equation}
	A_j = 
	\begin{bmatrix}
		A_j^1 & \cdots & A_j^D
	\end{bmatrix}
\end{equation}

\begin{equation}
A = 
\begin{bmatrix}
	A_1^T \\ A_2^T \\ \vdots \\ A_m^T
\end{bmatrix}
\hat{\bm{y}}= 
\begin{bmatrix}
	\hat{\bm{y}}_1^T \\ \hat{\bm{y}}_2^T \\ \vdots \\ \hat{\bm{y}}_m^T
\end{bmatrix}
\hat{\bm{f}}= 
\begin{bmatrix}
	(\hat{\bm{f}}^1)^T \\ (\hat{\bm{f}}^2)^T \\ \vdots \\ (\hat{\bm{f}}^D)^T
\end{bmatrix}
\end{equation}

\begin{equation}
	(A^H \Gamma A + W^H W) \hat{\bm{f}} = A^H \Gamma \hat{\bm{y}}
\end{equation}
%=======================================
\section{ECO \citep{DanelljanCVPR2017}} 
%=======================================
\section{Some tips of transfer from matlab to c++}
\begin{itemize}
	\item Debug tricks, to show the line number, file number and functions:
		\begin{lstlisting}[language=C++]
#define debug(a, args...) printf("%s(%s:%d) " a "\n", 
	__func__, __FILE__, __LINE__, ##args)
#define ddebug(a, args...) printf("%s(%s:%d) " a "\n", 
	__func__, __FILE__, __LINE__, ##args)
		\end{lstlisting}
	\item Debug tricks, to show the property of cv::mat:
		\begin{lstlisting}[language=C++]
void imgInfo(cv::Mat mat)
{
	int type = mat.type();
	string r;

	uchar depth = type & CV_MAT_DEPTH_MASK;
	uchar chans = 1 + (type >> CV_CN_SHIFT);

	switch (depth)
	{
	case CV_8U:
		r = "8U";
		break;
	case CV_8S:
		r = "8S";
		break;
	case CV_16U:
		r = "16U";
		break;
	case CV_16S:
		r = "16S";
		break;
	case CV_32S:
		r = "32S";
		break;
	case CV_32F:
		r = "32F";
		break;
	case CV_64F:
		r = "64F";
		break;
	default:
		r = "User";
		break;
	}

	r += "C";
	r += (chans + '0');

	debug("%s %d x %d", r.c_str(), mat.rows, mat.cols);
	//return r;
}
		\end{lstlisting}		
	\item Std functions better to name out to prevent default link to other functions. 
	For example: 
		\begin{lstlisting}[language=C++]
void absTest()
{
	std::vector<float> v{0.1, 0.2};
	float abs = abs(1.23f);
	debug("False result:%f", abs);

	abs = std::abs(1.23f);
	debug("True result:%f", abs);
}
		\end{lstlisting}
	For the False result, it sometimes link to other library and give answer 1.
	\item Take care of the parameter types, better to write it out clearly.
		\begin{lstlisting}[language=C++]
void accumulateTest()
{
	std::vector<float> v{0.1, 0.2};
	float sum = std::accumulate(v.begin(), v.end(), 0);
	debug("False result:%f", sum);

	sum = std::accumulate(v.begin(), v.end(), 0.0f);
	debug("True result:%f", sum);
}
		\end{lstlisting}
	For the False result, it gave answer 0, while the True result is 0.3.
	\item In opencv, color order: BGR, in matlab: RGB.
	\item In opencv mat, the data store in this order: $channel -> x -> y$. Test example:
		\begin{lstlisting}[language=C++]
void opencvTest()
{
	float *newdata = (float *)malloc(sizeof(float) * (2 * 3 * 4));
	for (int i = 0; i < 2 * 3 * 4; i++)
	{
		newdata[i] = i;
	}

	cv::Mat mat = cv::Mat(cv::Size(3, 4), CV_32FC(2), newdata);

	printf("\nInfo of original mat:");
	imgInfo(mat);
	for (int i = 0; i < 2 * 3 * 4; i++)
	{
		printf("%f ", mat.at<float>(0, i));
	}
	printf("\n");

	std::vector<cv::Mat> splitmat;
	cv::split(mat, splitmat);

	printf("\nInfo of splited mat:");
	imgInfo(splitmat[0]);

	printf("channel 0:\n");
	for (int j = 0; j < mat.rows; j++)
	{
		for (int i = 0; i < mat.cols; i++)
		{
			printf("%f ", splitmat[0].at<float>(j, i));
		}
		printf("\n");
	}
	printf("\n");
	printf("channel 1:\n");
	for (int j = 0; j < mat.rows; j++)
	{
		for (int i = 0; i < mat.cols; i++)
		{
			printf("%f ", splitmat[1].at<float>(j, i));
		}
		printf("\n");
	}
}
		\end{lstlisting}
	\item When using libcaffe.so, using the BLVC caffe version. I used the Nvidia version and GPU 
	not working, wasted my days.
\end{itemize}
%=======================================
\renewcommand\refname{Reference}
\bibliographystyle{unsrtnat}
\bibliography{VisualTracking} 
\clearpage
\end{document}