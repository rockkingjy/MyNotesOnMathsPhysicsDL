%\documentclass[12pt,draft]{article}
\documentclass[12pt]{article}
\usepackage{CJK}
\AtBeginDvi{\input{zhwinfonts}}
\usepackage{mathrsfs}
\usepackage{amsmath,amsthm,amsfonts,amssymb}
\usepackage{geometry}
\usepackage{fancyhdr}
\usepackage{indentfirst}
\usepackage{float}
\usepackage[dvips]{graphicx}
\usepackage{subfigure}
\usepackage[font=small]{caption}
\usepackage{threeparttable}
\usepackage{cases}
\usepackage{multicol}
\usepackage{url}
\usepackage{amsmath}
\usepackage{bm}
\usepackage{xcolor}
\usepackage{overpic}
\usepackage{natbib}
\usepackage[utf8]{inputenc}
\usepackage[american]{babel}
\usepackage{natbib}
\usepackage{graphicx}
\numberwithin{equation}{section}
\geometry{left=1.5cm,right=1.5cm,top=1.5cm,bottom=1.5cm}
\setlength{\parskip}{0.3\baselineskip}
\setlength{\headheight}{15pt}
\usepackage{color}   %May be necessary if you want to color links
\usepackage{hyperref}
\hypersetup{
    colorlinks=true, %set true if you want colored links
    linktoc=all,     %set to all if you want both sections and subsections linked
    linkcolor=blue,  %choose some color if you want links to stand out
}
\setlength{\parindent}{4ex}
\begin{document}\small
\title{Physics Notes}
\author{Yan JIN}
\pagestyle{fancy}\fancyhf{}
\lhead{}\rhead{JIN Yan}
\lfoot{\textit{}}\cfoot{}\rfoot{\thepage}
\renewcommand{\headrulewidth}{1.pt}
\renewcommand{\footrulewidth}{1.pt}
\maketitle
\tableofcontents
%=======================================
\section{Principles of Quantum Mechanics}
\subsection{Recipe to comprehend and command Quantum Mechanics: Deductive Reasoning and  Paradigm Shifts}
\subsubsection{Brief history of quantum physics}
\begin{enumerate}
\item Experimental discoveries leading to quantum mechanics \par
	\begin{enumerate}
	\item Discovery of \textbf{electron} - 1896, Thomson \par
		Electron, like quark, is a basic scientific concept that \textbf{we believe}! No one has ever seen an electron!
	\item Discovery of \textbf{Black-body radiation} - 1860, Kirchhoff
	\item Discovery of \textbf{Photoelectric effect} - 1887, Hertz
	\item Discovery of \textbf{Electromagnetic wave} - 1888, Hertz
	\end{enumerate}	
\item Theoretical innovations in quantum era
	\begin{enumerate}
	\item 1900 Plank equation, energy quantization hypothesis;
	\item 1905 Einstein light quanta hypothesis - rejected at first and accepted in 1919, with Millikan's detailed experiments on the photoelectric effect and the measurement of Compton scattering;
	\item 1925 Heisenberg, Born and Jordan formulate matrix mechanics;
	\item 1925 Dirac introduces canonical quantization, Heisenberg equation, ladder operators;
	\item 1925 Pauli states the quantum exclusion principle;
	\item 1925 Uhlenbeck and Goudsmit postulate spin;
	\item 1926 Schro\"{o}dinger formulates wave mechanics;
	\item 1926 Fermi discovers the spin-statistics connection;
	\item 1926 Dirac introduces superposition principle, Slater determinant, Fermi-Dirac statistics;
	\item 1927 Born interprets the probabilistic nature of wave functions;
	\item 1927 Heisenberg states the quantum uncertainty relation;
	\item 1927 Dirac introduces the Transformation theory;
	\item 1927 Dirac introduces the quantum electrodynamics;
	\item 1928 Dirac states Dirac equation - the fundamental equation of electron;
	\item 1930 Dirac introduces Dirac sea and hole theory;
	\item 1930 Dirac publishes \textit{Principles of Quantum Mechanics} to complete the quantum mechanical system;
	\item 1931 Dirac predicts the existence of positron;
	\item 1933 The completion of Quantum Mechanics was declared by two Nobel Prizes due to the discovery of positron as the experimental verification of Dirac electron theory.
	\end{enumerate}
\end{enumerate}
\subsubsection{Scientific Method}
\begin{enumerate}
\item Definition of science as given by Einstein
\item Greek philosophers: From Thales to Aristotle
\item Euclidean geometry and formal logical system
\item Renaissance and scientific revolution: From Copernicus to Newton
\item Descartes' Method of Science: The four precepts
\item The Cartesian geometry
\end{enumerate}
\subsubsection{Review of Classical Mechanics}
\begin{enumerate}
\item Newton's Principia
\item Einstein's critical review of Newtonian mechanics
	Four axioms:
	\begin{enumerate}
	\item Mass point hypothesis: the motion of an object can be depicted by Descartes' coordinators;
	\item Axiom of inertial;
	\item Axiom of motion of mass point;
	\item Axiom of interaction and anti-interaction between mass points;
	\end{enumerate}
\item From Newton to Hamilton
	Kant's two kinds of judgments:
	\begin{enumerate}
	\item \textbf{Analytic judgment}: a judgment whose truth can be determined just by analyzing the term in it;
	\item \textbf{Synthetic judgment}: a judgment whose truth requires looking outside of its terms.
	\end{enumerate}
	Kant said: The judgment of Euclidean geometry is \textbf{a priori synthetic}.\par
	Newton's Principia is \textbf{a priori synthetic}.\par
	Analytic Mechanics is \textbf{analytic judgment.}
\item The Lagrangian mechanics\par
	\textbf{Fermat's principle(or principle of least time)} has served as a guiding principle in the formulation of physical laws with the use of variational calculus. \par
	{\color{red}\textbf{Principle of least action}}: in all natural phenomena a quantity called \textbf{action} tends to be minimized. \par
	The action functional: 
	\begin{equation}
		S[q]=\int_{t_1}^{t_2}Ldt
	\end{equation}
	\begin{equation}
		L=L(q(t),\dot{q}(t))=\frac{m\dot{q}^2}{2}-V(q)
	\end{equation}
	\begin{equation}
		\frac{\delta S[q]}{\delta q(t)} = 0
	\end{equation}
	\begin{equation} \begin{aligned}
		\delta S =& \int_{t_1}^{t_2}dt[L(q+\delta q, \dot{q}+\delta \dot{q})-L(q,\dot{q})] \\
		=&\int_{t_1}^{t_2}dt[\delta q \frac{\partial L}{\partial q}+\delta \dot{q} \frac{\partial L}{\partial \dot{q}}] \\
		=&\int_{t_1}^{t_2}dt\delta q \frac{\partial L}{\partial q}+\int_{t_1}^{t_2}d(\delta q) \frac{\partial L}{\partial \dot{q}} \\
		=&[\delta q \frac{\partial L}{\partial \dot{q}}]_{t_1}^{t_2}+\int_{t_1}^{t_2}dt(\frac{\partial L}{\partial q}-\frac{d}{dt}\frac{\partial L}{\partial \dot{q}}) \\
		=&0
	\end{aligned} \end{equation}
	{\color{blue}Euler-Lagrange equation}:
	\begin{equation}
		\frac{\partial L}{\partial q}-\frac{d}{dt}\frac{\partial L}{\partial \dot{q}}=0
	\end{equation}
\item The Hamiltonian mechanics \par
	Legendre transformation:
	\begin{equation}
		H=\dot{q}\frac{\partial L}{\partial \dot{q}}-L=\dot{q}p-L=\frac{p^2}{2m}+V(q)=H(q, p)
	\end{equation}
	Hamilton equation:
	\begin{equation}\begin{aligned}
		\frac{dq}{dt}&=\frac{\partial H}{\partial p},\\
		\frac{dp}{dt}&=-\frac{\partial H}{\partial q}.
	\end{aligned}\end{equation}
	Poisson bracket:
	\begin{equation}
		\{A, B\}=\frac{\partial A}{\partial q}\frac{\partial B}{\partial p} -\frac{\partial A}{\partial p}\frac{\partial B}{\partial q} 
	\end{equation}
	{\color{blue}Hamilton equation again}:
	\begin{equation}\begin{aligned}
		\frac{dq}{dt}&=\{q, H\},\\
		\frac{dp}{dt}&=\{p, H\}.
	\end{aligned}\end{equation}
\end{enumerate}
\subsubsection{Paradigm in science and paradigm shifts in scientific revolutions}
\begin{enumerate}
\item Paradigm in science \par
	Thomas Kuhn
\item Paradigm shifts in scientific revolutions
\end{enumerate}
\subsubsection{Paradigm shifts: the recipe to comprehend and command Quantum Mechanics}
\begin{enumerate}
\item Paradigm shifts in theoretical quantum physics \par
	\begin{enumerate}
     	\item From {\color{blue}c-number} to {\color{red}q-number} \par
		{\color{blue} c-number Hamiltonian:
    		\[H=\frac{p^2}{2m}+\frac{1}{2}m\omega^2q^2\]} \par
		{\color{red} q-number Hamiltonian:
		\[\hat{H}=\frac{\hat{p}^2}{2m}+\frac{1}{2}m\omega^2\hat{q}^2\]} \par
	\item From {\color{blue}Poisson Bracket} to {\color{red}Quantum Bracket} \par
		The general quantization rule derived by Dirac in 1925:
		\begin{equation}\begin{split}
			{\color{red}[\hat{u},\hat{v}]}&={\color{blue}i\hbar \{u, v\}}, \\
			{\color{red}Quantum\ bracket} &= {\color{blue}Poisson\ bracket}
		\end{split}\end{equation} \par
		Dirac canonical quantization:
		\begin{equation}
			[{\color{red}\hat{q},\hat{p}]}={\color{blue}i\hbar \{q, p\}=i\hbar}
		\end{equation}
		{\color{blue} Hamilton equation of motion:
		\begin{equation}
			\frac{dq}{dt}=\{q, H\}
		\end{equation}} \par
		{\color{red}Quantum equation of motion - Heisenberg equation of motion(in Heisenberg picture):
		\begin{equation}
			\frac{d\hat{q}}{dt}=\frac{1}{i\hbar}[\hat{q}, \hat{H}]
		\end{equation}} \par
	\item From {\color{blue}visible physical space} to {\color{red}abstract mathematical space} \par
		{\color{blue}Classical mechanics: Euclidean space} $\leftrightarrow$ {\color{red}Quantum mechanics: Hilbert(Banach) space} \par
		{\color{blue}Classical mechanics: Mass point} $\leftrightarrow$ {\color{red}Quantum mechanics: State vector}

	\end{enumerate}
	===============================\par
	Dirac ladder operators:
	\begin{equation}\begin{split}
		\hat{a}^\dag=&\frac{\hat{p}}{\sqrt{2\hbar m\omega}}+i\sqrt{\frac{m\omega}{2\hbar}}\hat{q},\\
		\hat{a}=&\frac{\hat{p}}{\sqrt{2\hbar m\omega}}-i\sqrt{\frac{m\omega}{2\hbar}}\hat{q}
	\end{split}\end{equation}
	\begin{equation}
		\hat{a}^\dag\hat{a}
		=\frac{1}{\hbar\omega}(\frac{\hat{p}^2}{2m}+\frac{1}{2}m\omega^2\hat{q}^2)-\frac{1}{2}
	\end{equation}
	Planck oscillator is written as:
	\begin{equation}
		\hat{H}=(\hat{a}^\dag\hat{a}+\frac{1}{2})\hbar\omega
	\end{equation}
	\begin{equation}
		[\hat{a},\hat{a}^\dag]=1,[\hat{a}^\dag\hat{a},\hat{a}^\dag]=\hat{a}^\dag,
	 	[\hat{a}^\dag\hat{a},\hat{a}]=-\hat{a}
	 \end{equation} 
	 For vector $|n\rangle$ in Hilbert space:
	 \begin{equation}\begin{split}
	 	\hat{H}|n\rangle=&E_n|n\rangle \\
	 	\hat{a}^\pm|n\rangle=&C_\pm|n\pm1\rangle \\
		 \hat{H}(\hat{a}^\pm|n\rangle)=&(E_n\pm\hbar\omega)(\hat{a}^\pm|n\rangle)
	 \end{split}\end{equation}
	 where $\hat{a}^-=\hat{a}$, $\hat{a}^+=\hat{a}^\dag$.
\item Paradigm shifts in experimental quantum physics: \par
	The Stern-Gerlach experiment was performed in Frankfurt, Germany in 1922. The separation into distinct orbits was referred to as space-quantization between 1922-1927. However the correct explanation of this experiment was only given in 1927 when the atomic structure of Silver(Ag,Z=47)was clarified with one single electron outside filled shells. It's {\color{orange}the first direct experimental evidence of the electron spin}. And the direct observation of the spin is {\color{orange}the most direct proof of Quantization} in quantum mechanics. So the Stern-Gerlach experiment has become a {\color{orange}paradigm of quantum measurement}. \par
	Pauli matrix from two-level system (ref. Sakurai)
\end{enumerate}
%----------------------------------------------------
\subsection{Dirac's four axioms of Quantum Mechanics}2013-10-11
\subsubsection{Axiom I: Principle of superposition}
\begin{enumerate}
\item Definition of {\color{red}quantum states} \par
	In order to give {\color{orange}an absolute meaning to size}, we have to assume that there is {\color{orange}a limit to the fineness of our powers of observation} and the smallness of the accompanying disturbance - a limit which is inherent in the nature of things and can never be surpassed by improved technique or increased skill on the part of the observer. \par
	If the object under observation is such that the unavoidable limiting disturbance is negligible, then the object is big in the absolute sense and {\color{orange}we may apply classical mechanics} to it. \par
	If, on the other hand, the limiting disturbance is not negligible, then the object is small in the absolute sense and {\color{orange}we require a new theory} for dealing with it. \par
	A consequence of the preceding discussion is that we must revise our ideas of causality. {\color{orange}Causality applies only to a system which is left undisturbed}. If a system is small, we cannot observe it without producing a serious disturbance and hence we cannot expect to find any causal connection between the results of our observations. Causality will still be assumed to apply to undisturbed system and the equation which will be set up to describe an undisturbed system will be differential equations expressing a causal connection between conditions at one time and conditions at a later time. These equations will be in close correspondence with the equations of classical mechanics, but they will be connected only indirectly with the results of observation. There is an unavoidable indeterminacy in the calculation of observational results, the theory enabling us to calculate in general only the probability of our obtaining a particular result when we make an observation. \par
\item The general {\color{red}principle of superposition} \par
\item Mathematical formulation of the principle \par
	In mathematical physics, the Dirac-von Neumann axioms give a mathematical formulation of quantum mechanics in terms of operators on a Hilbert space. \par
\item Dirac's notation for vectors: the {\color{red}ket vectors} \par
	It is desirable to have a special name for describing the vectors which are connected with the states of a system in quantum mechanics. We shall call them ket vectors, or simply kets, denoted $|ket \rangle$. \par
	Linear dependency:
	\begin{equation}
		c_1|A\rangle+c_2|B\rangle=|C\rangle
	\end{equation}
	where $c_1$, $c_2$ are complex number. \par
	We now assume that each state of a dynamical system at a particular time corresponds to a ket vector, the correspondence being such that if a state results from the superposition of certain other states, its corresponding ket vector is expressible linearly in terms of the corresponding ket vectors of the other states, and conversely.
\item Dirac's introduction of inner product function and {\color{red}bra vectors} 
	\begin{equation}\begin{split}
		\langle B|(c|A\rangle) &= c\langle B|A\rangle, \\
		\langle B|(|A\rangle+|A'\rangle) &= \langle B|A\rangle + \langle B|A'\rangle
	\end{split}\end{equation}
\item The dual relationship between ket and bra \par
	We make the assumption that there is a one-one correspondence between the bras and kets:
	\begin{equation}
		c_1|A\rangle + c_2|B\rangle \leftrightarrow c_1^*\langle A| + c_2^*\langle B|
	\end{equation}
	Axioms of inner product:
	\begin{equation}\begin{split}
		\langle A|B \rangle &= \langle B|A \rangle^*, \\
		\langle A|A \rangle &> 0 
	\end{split}\end{equation}
	If 
	\begin{equation}\begin{split}
		\langle A|B \rangle = 0
	\end{split}\end{equation}
	we call A and B orthogonal.
\end{enumerate}
\subsubsection{Axiom II: Principle of observables}
\begin{enumerate}
\item {\color{red}Linear operators (q-numbers)} \par
	\begin{equation}\begin{split}
		\hat{O}|(c|A\rangle) &= c\hat{O}|A\rangle, \\
		\hat{O}|(|A\rangle+|A'\rangle) &= \hat{O}|A\rangle + \hat{O}|A'\rangle
	\end{split}\end{equation}
\item Operator operating on the bra vectors \par
	\begin{equation}
		\langle B'| = \langle B|\hat{O}
	\end{equation}
	We now have a complete algebraic scheme involving three kinds of quantities, {\color{red}bra vectors}, {\color{red}ket vectors} and {\color{red}linear operators}. They can be multiplied together in the various ways discussed above, and the associative and distributive axioms of multiplication always hold, but the commutative axiom of multiplication does not hold.
\item Conjugate relations \par
	Adjoint operator $\hat{O}^\dag$:
	\begin{equation}
		\hat{O} |A\rangle \leftrightarrow \langle A|\hat{O}^\dag
	\end{equation}
	\begin{equation}
		\langle A|\hat{O}^\dag|B\rangle = \langle B |\hat{O} |A\rangle^*
	\end{equation}
	If $\hat{H}^\dag=\hat{H}$, then we call $\hat{H}$ as \textbf{self-adjoint operator} or {\color{red}\textbf{Hermitian operator}}.
\item Eigenvalues and eigenvectors
	\begin{equation}
		\hat{\alpha} |\alpha \rangle = \alpha|\alpha \rangle
	\end{equation}
\item The eigenvalue problem of Hermitian operators \par
	Two properties of Hermitian operator $\hat{H}$:
	\begin{enumerate}
		\item $\hat{H}$'s eigenvalues are real;
		\item $\hat{H}$'s eigenvectors for different eigenvalues are orthogonal.
	\end{enumerate}
\item {\color{red}Axioms of observables} in quantum mechanics: \par
	The connections between Quantum Theory and Experiments:
	\begin{enumerate}
		\item Hermitian operators: \textbf{Observables};  
		\item Eigenvectors: The states of the system;
		\item Corresponding eigenvalues: The measurements' values from experiments.
	\end{enumerate}
	Explanation of the Stern-Gerlach experiment: \par
	The separation into two distinct orbits in SGz apparatus is caused by the two eigenstates of $\hat{S}_z$:
	\begin{equation}
		|\uparrow \rangle = |+\frac{\hbar}{2}\rangle=\begin{bmatrix}1\\0\end{bmatrix},
		|\downarrow \rangle = |-\frac{\hbar}{2}\rangle=\begin{bmatrix}0\\1\end{bmatrix}
	\end{equation}
	\begin{equation}
		\hat{S}_z=\begin{bmatrix}\frac{\hbar}{2}&0\\0&-\frac{\hbar}{2}\end{bmatrix}
	\end{equation}
	The magnetic fields in Stern-Gerlach experiment:
	\begin{equation}
		\bm{B}(z)=[0,0,B(z)]
	\end{equation}
	So with the $\hat{\bm{\mu}}=-g\mu_B\hat{\bm{S}}/\hbar$, we have:
	\begin{equation}
		\hat{H}=-\hat{\bm{\mu}}\cdot\bm{B}(z)=g\mu_BB(z)\hat{S_z}/\hbar
	\end{equation}
	{\color{orange}Before Ag atom goes into the magnetic fields}, it is in the state of superposition of two eigenstates:
	\begin{equation}
		|Ag \rangle = c_+|+\frac{\hbar}{2}\rangle+ c_-|-\frac{\hbar}{2}\rangle
	\end{equation}
	{\color{orange}When Ag atom just goes into the magnetic fields}, it collapse into one of the eigenstates, and accelerate as the eigenstate it jumped into by the magnetic fields.\par
	{\color{orange}After Ag atom goes out of the magnetic fields}, it keeps the speed and direction.
\end{enumerate}
\subsubsection{Axiom III: Quantization conditions}
\begin{enumerate}
\item Sequential Stern-Gerlach experiment again
\item Commutability and compatibility \par
	The commutable operators are compatible(they should have the same eigenvector). \par
	The incommutable Hermitian operators are incompatible(they have no complete eigenvectors).
\item Uncertainty relation between incompatible observables \par
	The mean value of Hermitian operator $\hat{u}$ for state $|A\rangle$ is: $\langle\hat{u}\rangle=\langle A|\hat{u}|A\rangle$; \par
	Define: $\Delta \hat{u}=\hat{u}-\langle\hat{u}\rangle$; \par
	Mean square deviation: $\langle(\Delta \hat{u})^2\rangle=\langle(\hat{u}-\langle\hat{u}\rangle)^2\rangle=\langle\hat{u}^2\rangle-\langle\hat{u}\rangle^2$; \par
	Uncertainty relation: for any two Hermitian operator $\hat{u}$ and $\hat{v}$: 
	\begin{equation}
		\langle (\Delta \hat{u})^2\rangle\langle(\Delta \hat{v})^2\rangle 
		\geq \frac{1}{4}|\langle[\hat{u},\hat{v}]\rangle|^2.
	\end{equation}
\item {\color{red}Axiom of quantization conditions}: Dirac canonical quantization \par
	The general quantization rule derived by Dirac in 1925:
	\begin{equation}
		{\color{red}[\hat{u},\hat{v}]=i\hbar \{u, v\}}
	\end{equation}
	Dirac canonical quantization:
	\begin{equation}
		[\hat{q},\hat{p}]=i\hbar \{q, p\}=i\hbar
	\end{equation}
\item Heisenberg uncertainty relation between x and p \par
	For x and p, define root mean square deviation:
	\begin{equation}
		\Delta x=\sqrt{\langle(\Delta \hat{x})^2 \rangle}, 
		\Delta p=\sqrt{\langle(\Delta \hat{p})^2 \rangle}
	\end{equation}
	Then we have:
	\begin{equation}
		\Delta x \Delta p \geq \frac{\hbar}{2}
	\end{equation}
\item Uncertainty between x and p in the eigenenergy states of Planck oscillator \par
	Dirac ladder operators:
	\begin{equation}\begin{split}
		\hat{a}^\dag=&\frac{\hat{p}}{\sqrt{2\hbar m\omega}}+i\sqrt{\frac{m\omega}{2\hbar}}\hat{q},\\
		\hat{a}=&\frac{\hat{p}}{\sqrt{2\hbar m\omega}}-i\sqrt{\frac{m\omega}{2\hbar}}\hat{q}
	\end{split}\end{equation}
	Then we have:
	\[2i\sqrt{\frac{m\omega}{2\hbar}}\langle n|\hat{q}|n\rangle=\langle n|(\hat{a}^\dag-\hat{a})|n\rangle=0\]
	\[(2i\sqrt{\frac{m\omega}{2\hbar}})^2\langle n|\hat{q}^2|n\rangle
	=\langle n|(\hat{a}^\dag-\hat{a})^2|n\rangle
	=\langle n|(\hat{a}^\dag)^2+\hat{a}^2-\hat{a}^{\dag} \hat{a}-\hat{a}\hat{a}^{\dag}|n\rangle
	=\langle n|-2\hat{a}^\dag\hat{a}-1|n \rangle=-2n-1\]
	So:
	\[(2i\sqrt{\frac{m\omega}{2\hbar}})^2(\langle n|\hat{q}^2|n\rangle-\langle n|\hat{q}|n\rangle^2)=-(2n+1)\]
	And:
	\[\frac{2}{\sqrt{2\hbar m\omega}}\langle n|\hat{p}|n\rangle=\langle n|(\hat{a}^\dag-\hat{a})|n\rangle=0\]
	\[(\frac{2}{\sqrt{2\hbar m\omega}})^2\langle n|\hat{p}^2|n\rangle
	=\langle n|(\hat{a}^\dag+\hat{a})^2|n\rangle
	=\langle n|(\hat{a}^\dag)^2+\hat{a}^2+\hat{a}^{\dag} \hat{a}+\hat{a}\hat{a}^{\dag}|n\rangle
	=\langle n|2\hat{a}^\dag\hat{a}+1|n \rangle=2n+1\]
	So:
	\[(\frac{2}{\sqrt{2\hbar m\omega}})^2(\langle n|\hat{p}^2|n\rangle-\langle n|\hat{p}|n\rangle^2)=2n+1\]
	Then:
	\[(\Delta x)^2(\Delta p)^2=\langle(\Delta \hat{x})^2 \rangle \langle(\Delta \hat{p})^2 \rangle=-(2n+1)^2(2i\sqrt{\frac{m\omega}{2\hbar}})^{-2}(\frac{2}{\sqrt{2\hbar m\omega}})^{-2}=(n+\frac{1}{2})^2\hbar^2\]
	At last we have the uncertainty between x and p in the eigenenergy states of Planck oscillator:
	\begin{equation}
		\Delta x\Delta p=(n+\frac{1}{2})\hbar
	\end{equation}
\end{enumerate}
\subsubsection{Axiom IV: Equation of motion}
\begin{enumerate}
\item The Heisenberg equation of motion \par
	The states are fixed: 
	\begin{equation}
		\frac{d\hat{\alpha}(t)}{dt}=\frac{1}{i\hbar}[\hat{\alpha}(t), \hat{H}(t)]
	\end{equation}
\item The Schr\"{o}dinger equation of motion \par
	The operator are fixed:
	\begin{equation}
		i\hbar\frac{d|A, t\rangle}{dt}=\hat{H}(t)|A, t\rangle
	\end{equation}
\end{enumerate}
%----------------------------------------------------
\subsection{Dirac's three rules of manipulations in Quantum Mechanics: 
                       Representations, Transformations and Pictures}
         2013-10-29 \par
	Representations: digitization; \par
	Transformation: symmetry; \par   
	Pictures: time evolution.                               
\subsubsection{Representations of discrete eigenvalue spectra - matrix mechanics}
\begin{enumerate}
\item The basis of a linear vector space and the basis vectors
\item The eigenvectors of Hermitian operators as orthonormal basis of Hilbert space
\item The discrete eigenvalue spectra and the matrix representation or matrix mechanics
\item Matrix (energy or Heisenberg) representation of Planck oscillator
\item Matrix representation of spin one half and the Stern-Gerlach experiment again
\end{enumerate}
\subsubsection{Transformations}
\begin{enumerate}
\item Spin operators along arbitrary direction and rotation of the basis vectors
\item Transformation of representations of discrete eigenvalue spectra
\item Infinitesimal spatial displacement and momentum operator
\item Finite translation as unitary transformation
\item Properties of unitary operator, infinitesimal unitary transformation and Hermitian operator, symmetry and invariance
\end{enumerate}
\subsubsection{Representations of continuous eigenvalue spectra}
\begin{enumerate}
\item Normalization of basis eigenvectors of the abstract linear vector space
\item Dirac delta function
\end{enumerate}
\subsubsection{The coordinate representation and the wave function}
\begin{enumerate}
\item The basis and the wave function
\item The coordinate representation of momentum operator
\item The "matrix elements" of the canonical quantization condition in the coordinate representation
\item The eigenfunction of momentum operator
\end{enumerate}
\subsubsection{Derivation of stationary state Schr\"{o}dinger's wave equation}
\begin{enumerate}
\item The non-relativistic Hamiltonian
\item The eigenequation of the Hamiltonian
\item The coordinate representation of the eigenequation of the Hamiltonian and
    the stationary state Schr\"{o}dinger's wave equation
\end{enumerate}
\subsubsection{Solution of eigenfunctions of Planck oscillator}
\begin{enumerate}
\item The wave function of the ground state   
\item The wave function of all eigenstates      
\end{enumerate}
\subsubsection{Representations of mixing discrete and continuous eigenvalue spectra}
\subsubsection{Complete set of dynamical quantities and simultaneous measurement of compatible observables}	
\subsubsection{Pictures and Axiom IV of quantum mechanics-quantum dynamics}
	Heisenberg Representation(Energy Representation): eigenvectors of Hamilton operator as basis vectors; \par
	Schr\"{o}dinger Representation(Coordinate Representation): eigenvectors of Coordinate operator as basis vectors.
\begin{enumerate}
\item The time evolution of quantum states
	\begin{equation}
		|A, t\rangle = \hat{T}|A,t_0\rangle
	\end{equation}
	where $\hat{T}$ is a time revolution operator. \par
	Time is not a observer, because time is not a eigenvalue of any Hermitian Operator. \par
	Assumption: 
	\begin{equation}
		\langle A, t|A, t\rangle = \langle A, t_0|A, t_0\rangle
	\end{equation}
	Then:
	\begin{equation}
		\hat{T}^\dag\hat{T} = 1, 
		\hat{T}^{-1} = \hat{T}^\dag
	\end{equation}
\item The Schr\"{o}dinger equation of motion of quantum states
	\begin{equation}
		i\hbar\frac{d|A, t\rangle}{dt} = \hat{H}(t)|A, t\rangle
	\end{equation}
\item The Schr\"{o}dinger picture \par
	Ket: not fixed, operator: fixed.
\item Derivation of the time-dependent Schr\"{o}dinger wave equation
\item The stationary state 
\item The Heisenberg picture \par
	Ket: fixed, operator: not fixed. \par
	Here we denote Schr\"{o}dinger picture as:
	\begin{equation} \begin{split}
		|A, t\rangle_S &= |A, t\rangle = \hat{T}(t, t_0)|A,t_0\rangle \\
		\hat{\alpha}_S(t) &= \hat{\alpha}(t_0) = \hat{\alpha}
	\end{split} \end{equation}
	For Heisenberg picture:
	\begin{equation} \begin{split}
		|A, t\rangle_H &= \hat{T}^{-1}(t, t_0)|A, t\rangle_S \\
		\hat{\alpha}_H(t) &= \hat{T}^{-1}(t, t_0)\hat{\alpha}_S(t)\hat{T}(t, t_0)
	\end{split} \end{equation}
	Heisenberg equation of motion:
	\begin{equation}
		\frac{d\hat{\alpha}_H(t)}{dt} = \frac{1}{i\hbar}[\hat{\alpha}_H(t), \hat{H}(t)]
	\end{equation}
\item The Heisenberg equation of motion of Planck oscillator
\item The Ehrenfest theorem
\item Conservation laws in quantum mechanics
\end{enumerate}
\subsubsection{Coherent states of Planck oscillator}
\begin{enumerate}
\item The coherent state
\item The fourth Dirac ladder operators
\item Expansion in Heisenberg representation
\item Over-completeness
\item Non-orthogonality
\item The least uncertainty states in QM
\item The time-dependent coherent state
\item The coherent state Gaussian wave packet
\end{enumerate}
%----------------------------------------------------
\subsection{Dirac's theory of electron}
\subsubsection{Dirac equation}
2013-11-22
\begin{enumerate}
\item The special theory of relativity\par
	What is time?\par
	Two basic Axioms:
	\begin{enumerate}
		\item The Principle of Relativity\par
		The laws of physics are the same in all inertial frames of reference;
		\item The Constancy of Speed of Light in Vacuum\par
		The speed of light in vacuum has the same value c in all inertial frames of reference;			\end{enumerate}
\item The four vectors and mass-energy equivalence \par
	\begin{equation}x_1=x, x_2=y, x_3=z, x_4=ict;\end{equation}
	\begin{equation}x_\mu x_\mu = \sum_{\mu=1}^{4} x_\mu x_\mu = x^2-c^2 t^2; \end{equation}
	\begin{equation}p_1=p_x, p_2=p_y, p_3=p_z, p_4=i\frac{E}{c};\end{equation}
	\begin{equation}p_\mu p_\mu = \sum_{\mu=1}^{4} p_\mu p_\mu = p^2-\frac{E^2}{c^2};\end{equation}
	Lorenz transformation requires that $x_\mu x_\mu$ and $p_\mu p_\mu$ etc. remain constant.\par
	Now, choose a special coordinate with $p'=0$ and $E'=E_0$, then 
	$p_\mu p_\mu=p'_\mu p'_\mu=0-\frac{E_0^2}{c^2}$, and define $m_0=\frac{E_0^2}{c^2}$, 
	then we have the mass-energy equivalence:
	\begin{equation}
		E^2=c^2p^2+m_0^2c^4
	\end{equation} \par
	From the definition: $p=\frac{m_0v}{\sqrt{1-\frac{v^2}{c^2}}}$, we get(get rid of the minus one):
	\begin{equation}
		E=mc^2=\frac{m_0c^2}{\sqrt{1-\frac{v^2}{c^2}}}
	\end{equation}
\item The Klein-Gordon equation \par
	Replace mass-energy equivalence equation with:
	\begin{equation}
		E \rightarrow i\hbar \frac{\partial}{\partial{t}}, 
		\hat{\mathbf{p}} \rightarrow \frac{\hbar}{i} \hat{\bm{\nabla}}
	\end{equation}
	we get the Klein-Gordon equation:
	\begin{equation}
		-\hbar^2\frac{\partial^2{\psi(\mathbf{x},t)}}{\partial{t}^2}=
		(-\hbar^2 c^2\hat{\bm{\nabla}}^2+m^2_{0}c^4)\psi(\mathbf{x},t)
	\end{equation}
	use nature unit: $\hbar=c=1$, and introduce $\partial_{\mu}=\frac{\partial}{\partial{x}_{\mu}}$, 
	we have a more elegant form of this equation:
	\begin{equation}
		(\partial_{\mu}\partial_{\mu}-m^2_0)\psi(\mathbf{x},t)=0
	\end{equation}
	We can see that it's invariant under Lorentz transformations.\par
	Solve this equation with: 
	$\psi(\mathbf{x},t)=(2\pi\hbar)^{-\frac{3}{2}}e^{\frac{i}{\hbar} p_\mu x_\mu}=
	(2\pi\hbar)^{-\frac{3}{2}}e^{\frac{i}{\hbar} (\mathbf{p}\cdot\mathbf{x}-Et)}$, 
	we have:$E^2=c^2p^2+m_0^2c^4$ again, but that means the eigenvalues have two values:
	\begin{equation}
		E=\pm\sqrt{c^2p^2+m_0^2c^4}
	\end{equation}
	We have no reason to throw out the negative one, because all the two are the eigenvalues, so all of  them are physical values.\par
	 But {\color{orange}the big problem for Klein-Gordon equation is that it can not define Probabilistic Density}:\par
	The current in form of four-vectors:
	\begin{equation}j_\mu=(\mathbf{j}, i\rho)\end{equation}
	And it keeps the consistency equation:
	\begin{equation}\partial_\mu j_\mu=0\end{equation}
	In non-relativistic quantum mechanics, from Schr\"{o}dinger equation, the Probability Density is:
	\begin{equation}\rho=\psi^\dag\psi\end{equation}
	and the probability current is:
	\begin{equation}
		\mathbf{j}=\frac{i}{2m}(\psi\nabla\psi^\dag-\psi^\dag\nabla\psi)
	\end{equation}
	But here from Klein-Gordon equation:
	\begin{equation}
		\rho=i[\psi^\dag\frac{\partial \psi}{\partial t}-\psi\frac{\partial \psi^\dag}{\partial t}]
	\end{equation}
	\begin{equation}
		\mathbf{j}=i(\psi\nabla\psi^\dag-\psi^\dag\nabla\psi)
	\end{equation}
	we can see that {\color{orange}$\rho$ need not to be positive}.
\item Derivation of Dirac equation based on the four axioms of QM \par
	Now we need a equation that fits both Schr\"{o}dinger equation and Lorentz transformation.\par
	\begin{equation}
		\hat{H}=\sqrt{c^2\hat{p}^2+m_0^2c^4}
		=c\hat{\bm{\alpha}}\cdot\hat{\mathbf{p}}+\hat{\beta} m_0c^2
	\end{equation}
	that means:
	\begin{equation} 
		c^2\hat{p}^2+m_0^2c^4
		=(c\hat{\bm{\alpha}}\cdot\hat{\mathbf{p}}+\hat{\beta} m_0c^2)^2
	\end{equation}
	Then we have:
	\begin{equation}\begin{split}
		\hat{\alpha_1^2}=\hat{\alpha_2^2}=\hat{\alpha_3^2}=\hat{\beta^2}=1 \\
		\hat{\bm{\alpha}}\hat{\beta}+\hat{\beta}\hat{\bm{\alpha}}=0 \\
		\hat{\alpha_1}\hat{\alpha_2}+\hat{\alpha_2}\hat{\alpha_1}=0 \\
		\hat{\alpha_2}\hat{\alpha_3}+\hat{\alpha_3}\hat{\alpha_2}=0 \\
		\hat{\alpha_3}\hat{\alpha_1}+\hat{\alpha_1}\hat{\alpha_3}=0 \\
	\end{split}\end{equation}
	Replace this into the Schr\"{o}dinger equation of motion, then we have dirac equation of motion:
	\begin{equation}
		i\hbar\frac{d|A,t\rangle}{dt}=
		(c\hat{\bm{\alpha}}\cdot\hat{\mathbf{p}}+\hat{\beta} m_0c^2)|A,t\rangle
	\end{equation}
	For $\hat{\bm{\alpha}}$ and $\hat{\beta}$, we choose a 4-dimension vector space, and the base vectors are:
	\[|1\rangle=\begin{bmatrix}1\\ 0\\ 0\\ 0\\\end{bmatrix}, 
	|2\rangle=\begin{bmatrix}0\\ 1\\ 0\\ 0\\\end{bmatrix},
	|3\rangle=\begin{bmatrix}0\\ 0\\ 1\\ 0\\\end{bmatrix}, 
	|4\rangle=\begin{bmatrix}0\\ 0\\ 0\\ 1\\\end{bmatrix}\]
	To calculate the dirac equation of motion, we choose the base vectors as:
	\begin{equation}|x,y,z;n\rangle=|n\rangle|x,y,z\rangle\end{equation}
	where,
	\[|x,y,z;1\rangle=\begin{bmatrix}|x,y,z\rangle\\ 0\\ 0\\ 0\\ \end{bmatrix},
	|x,y,z;2\rangle=\begin{bmatrix}0\\ |x,y,z\rangle\\ 0\\ 0\\ \end{bmatrix},
	|x,y,z;3\rangle=\begin{bmatrix}0\\ 0\\ |x,y,z\rangle\\ 0\\ \end{bmatrix},
	|x,y,z;4\rangle=\begin{bmatrix}0\\ 0\\ 0\\ |x,y,z\rangle\\ \end{bmatrix}\]
	These base vectors are orthogonal and complete, so for each state vector of Dirac's electron $|A,t\rangle$:
	\[|A,t\rangle=\sum_{n=1}^4\iiint dxdydx |x,y,z;n\rangle\langle x,y,z;n|A,t\rangle=\sum_{n=1}^4\iiint dxdydx\psi_n(x,y,z,t) |x,y,z;n\rangle\]
	where,
	\[\psi_n(x,y,z,t)=\langle x,y,z;n|A,t\rangle\]
	Then the dirac equation of motion becomes :
	\begin{equation}
		i\hbar\frac{\partial{\psi}}{dt}=
		(c\hat{\bm{\alpha}}\cdot\frac{\hbar}{i}\hat{\bm{\nabla}}+\hat{\beta} m_0c^2)\psi
	\end{equation}
	where,
	\[\psi=\psi(x,y,z,t)=\begin{bmatrix}\psi_1(x,y,z,t) \\ \psi_2(x,y,z,t) \\ \psi_3(x,y,z,t) \\ \psi_4(x,y,z,t) \\\end{bmatrix}
	=\begin{bmatrix}\psi_1 \\ \psi_2 \\ \psi_3 \\ \psi_4 \end{bmatrix}\]
\item Playing with the Pauli matrices \par
	Let $\hat{\alpha}_4=\hat{\beta}$, we could rewrite the relations before:
	\begin{equation}
		\{\hat{\alpha}_\mu, \hat{\alpha}_\nu \}=
		\hat{\alpha}_\mu \hat{\alpha}_\nu + \hat{\alpha}_\nu \hat{\alpha}_\mu = 
		2 \delta_{\mu \nu}
	\end{equation}
	This is just like the relation between pauli matrix.\par
	Now we assume $m_0=0$, the static mass is zero:
	\begin{equation}
		\hat{H}=\sqrt{c^2\hat{p}^2+m_0^2c^4}
		=c\sqrt{\hat{p}^2}=c\sqrt{\hat{\mathbf{p}}\cdot\hat{\mathbf{p}}}
		=c\sqrt{(\hat{\bm{\sigma}}\cdot\hat{\mathbf{p}})^2}
		=c(\hat{\bm{\sigma}}\cdot\hat{\mathbf{p}})
	\end{equation}
	Replace it into the the Schr\"{o}dinger equation of motion:
	\begin{equation}
		i\hbar\frac{d|A,t\rangle}{dt}
		=c(\hat{\bm{\sigma}}\cdot\hat{\mathbf{p}})|A,t\rangle
	\end{equation}
	This is the equation for particles with spin 1/2 and zero static mass. If we constrain the $\hat{\mathbf{p}}=(\hat{p}_x,\hat{p}_y)$, then it's the 2010 Nobel Prize's work on graphene, the weyl equation. \\
	Pauli matrix:
	\begin{equation}
		\hat{\sigma}_x=\begin{bmatrix}0&1\\1&0\end{bmatrix},
		\hat{\sigma}_y=\begin{bmatrix}0&-i\\i&0\end{bmatrix},
		\hat{\sigma}_z=\begin{bmatrix}1&0\\0&-1\end{bmatrix}
	\end{equation}
	it has relation:
	\begin{equation}
		(\hat{\bm{\sigma}}\cdot\mathbf{a})(\hat{\bm{\sigma}}\cdot\mathbf{b})
		=\mathbf{a}\cdot\mathbf{b}+i\hat{\bm{\sigma}}\cdot(\mathbf{a}\times\mathbf{b})
	\end{equation}
\item Dirac equation in Dirac representation
	\begin{equation}
		\hat{\mathbf{\alpha}}=\begin{bmatrix}0&\hat{\bm{\sigma}}\\\hat{\bm{\sigma}}&0\end{bmatrix},
		\hat{\bm{\beta}}=\begin{bmatrix}\hat{\mathbf{1}}&0\\0&-\hat{\mathbf{1}}\end{bmatrix},
	\end{equation}
	so we have the Dirac equation in Dirac representation:
	\begin{equation}
		i\hbar\frac{\partial \psi}{\partial t}=
		[c\begin{bmatrix}0&\hat{\bm{\sigma}}\\\hat{\bm{\sigma}}&0\end{bmatrix}\cdot(\frac{\hbar}{i}		\hat{\bm{\nabla}})+\begin{bmatrix}\hat{\mathbf{1}}&0\\0&-\hat{\mathbf{1}}\end{bmatrix}m_0c^2]\psi
	\end{equation}
\item Dirac equation in Weyl representation
	\begin{equation}
		\hat{\mathbf{\alpha}}=\begin{bmatrix}\hat{\bm{\sigma}}&0\\0&-\hat{\bm{\sigma}}\end{bmatrix},
		\hat{\mathbf{\beta}}=\begin{bmatrix}0&\hat{\mathbf{1}}\\\hat{\mathbf{1}}&0\end{bmatrix},
	\end{equation}
	so we have the Dirac equation in Weyl representation:
	\begin{equation}
		i\hbar\frac{\partial \psi}{\partial t}=
		[c\begin{bmatrix}\hat{\bm{\sigma}}&0\\0&-\hat{\bm{\sigma}}\end{bmatrix}\cdot(\frac{\hbar}{i}		\hat{\bm{\nabla}})+\begin{bmatrix}0&\hat{\mathbf{1}}\\\hat{\mathbf{1}}&0\end{bmatrix}m_0c^2]\psi
	\end{equation}
	Mr. Nambu found the similarity between the Weyl representation and the BCS superconductivity theory, and do his Nobel Prize work.
\end{enumerate}
\subsubsection{The motion of a free electron}
2013-11-29
\begin{enumerate}
\item The velocity of Dirac electron and zitterbewegung
\item The current conservation of a Dirac electron
\item Conservation of angular momentum of a Dirac electron and spin
\item Complete set of dynamical quantities of Dirac electron and helicity
\item The plane wave solution of a free Dirac electron
\end{enumerate}
\subsubsection{Dirac sea, Dirac hole and Modern Concept of Particles}
\begin{enumerate}
\item The concept of Dirac sea based on the Pauli exclusion principle
\item The assumption about observables   
\item The concept of particles in QM
\item Dirac's hole theory, concept of anti-particle and prediction of positron
\item The electron-positron pair production
\end{enumerate}
\subsubsection{Quantum field theoretical formulation of Dirac hole theory}
\begin{enumerate}
\item The energy of many Dirac electrons as eigenenergy of a new Hamiltonian
\item The Fock representation
\item The quantum field theoretical Hamiltonian of electrons and positrons
\item The second quantization method
\end{enumerate}
\subsubsection{The non-relativistic limit of Dirac equation}
\begin{enumerate}
\item A Dirac electron in electromagnetic field
\item Eliminating the rest mass of electron
\item The Pauli equation
\item The magnetic moment of Dirac electron
\item The spin-orbit coupling
\end{enumerate}
\subsubsection{Covariance of Dirac equation under symmetrical transformations}
\begin{enumerate}
\item Covariance of Dirac equation under Lorentzian and rotational transformations
\item Dirac matrix, Dirac algebra and Dirac spinor
\item Lorentzian transformations
\item Rotational transformation around z axis
\item Properties of the coordinate transformation matrices
\item Coordinate transformation of Dirac equation
\item Direct transformation of Dirac spinor and Dirac matrices
\item Combined rotational transformations of Dirac spinor
\end{enumerate}
\subsubsection{Neutrino, Dirac Fermion, Photon}
\begin{enumerate}
\item Neutrino\par
	A neutrino is an electrically neutral, weakly interacting elementary subatomic particle with half-integer spin. The neutrino is denoted by the Greek letter $\nu$. All evidence suggests that neutrinos have mass but their mass is tiny even by the standards of subatomic particles. Their mass has never been measured accurately.\par
	The neutrino was postulated first by Wolfgang Pauli in 1930 to explain how beta decay could conserve energy, momentum, and angular momentum(spin).\par
	In quantum field theory, the Weyl equation is a relativistic wave equation for describing massless spin -1/2 particles:
	\begin{equation}
		i\hbar\frac{\partial \psi(\bm{r},t)}{\partial t}=
		c\bm{\sigma}\cdot(\frac{\hbar}{i}\hat{\bm{\nabla}})\psi(\bm{r},t)
	\end{equation}
	Assume the plan-wave: $\psi(\mathbf{r},t)=e^{\frac{i}{\hbar}(\bm{p} \cdot \bm{r}-Et)}\phi(\bm{p})$, replace it into Weyl equation:
	\begin{equation}
		c\bm{\sigma}\cdot\bm{p}\phi(\bm{p})=E\phi(\bm{p})
	\end{equation}
	Hamilton operator: $\hat{H}=c\bm{\sigma}\cdot\bm{p}$\par
	Spin operator in direction $\bm{n}=(sin\theta cos\phi, sin\theta sin\phi, cos\theta)$: 
	$\hat{S}_n=\frac{\hbar}{2}\bm{\sigma}\cdot\bm{n}
	=\frac{\hbar}{2}\frac{\bm{\sigma}\cdot\bm{p}}{p}=\frac{\hbar}{2}\hat{h}$ \par
	we have: $[\hat{H},\hat{h}]=0$, so we have a complete set of operators.\par
	solve the equation: 
	\begin{equation}\begin{split}
		E_1=cp, h=+1, \psi_1(\bm{p})&=\begin{bmatrix}cos\frac{\theta}{2}e^{-i\frac{\phi}{2}} \\ sin\frac{\theta}{2}e^{i\frac{\phi}{2}} \end{bmatrix}\\
		E_2=-cp, h=-1, \psi_2(\bm{p})&=\begin{bmatrix}-sin\frac{\theta}{2}e^{-i\frac{\phi}{2}} \\ cos\frac{\theta}{2}e^{i\frac{\phi}{2}} \end{bmatrix}
	\end{split}\end{equation}
	For both positive and negative energy solutions, only one kind of spin, the parity conservation is broken!
\item Dirac Fermion\par
	2010 Nobel Prize on Graphene
	2D-Dirac equation:
	\begin{equation}
		i\hbar\frac{\partial \psi(\bm{r},t)}{\partial t}=
		v\bm{\sigma}\cdot(\frac{\hbar}{i}\hat{\bm{\nabla}})\psi(\bm{r},t)
	\end{equation}
	where $\bm{r}=x\bm{e_x}+y\bm{e_y}$ is 2D vector, v is a constant.\par
	Assume the plan-wave: $\psi(\mathbf{r},t)=e^{\frac{i}{\hbar}(\bm{p} \cdot \bm{r}-Et)}\phi(\bm{p})$, replace it into 2D-Dirac equation:
	\begin{equation}
		v\bm{\sigma}\cdot\bm{p}\phi(\bm{p})=E\phi(\bm{p})
	\end{equation}
	we choose $\bm{p}=p_x\bm{e_x}+p_y\bm{e_y}=p(cos\theta\bm{e_x}+sin\theta\bm{e_y})$, then $\bm{\sigma}\cdot\bm{p}=p_x\sigma_x+p_y\sigma_y=p\begin{bmatrix}0&e^{-i\theta}\\e^{i\theta}&0\end{bmatrix}$\par
	Hamilton operator: $\hat{H}=v\bm{\sigma}\cdot\bm{p}$\par
	Spin operator in direction $\bm{n}=(sin\theta cos\phi, sin\theta sin\phi, cos\theta)$: 
	$\hat{S}_n=\frac{\hbar}{2}\bm{\sigma}\cdot\bm{n}
	=\frac{\hbar}{2}\frac{\bm{\sigma}\cdot\bm{p}}{p}=\frac{\hbar}{2}\hat{h}$ \par
	we have: $[\hat{H},\hat{h}]=0$, so we have a complete set of operators.\par
	solve the equation: 
	\begin{equation}\begin{split}
		E_1=vp, h=+1, \psi_1(\bm{p})&=\frac{1}{\sqrt{2}}\begin{bmatrix}e^{-i\theta} \\ 1\end{bmatrix}\\
		E_2=-vp, h=-1, \psi_2(\bm{p})&=\frac{1}{\sqrt{2}}\begin{bmatrix}1 \\ -e^{i\theta}\end{bmatrix}
	\end{split}\end{equation}
\item Spin of photon\par
	Maxwell equation in vacuum:
	\begin{equation}
		\bm{\nabla}\times\bm{E}=-\frac{1}{c}\frac{\partial\bm{B}}{\partial t},
		\bm{\nabla}\times\bm{B}=\frac{1}{c}\frac{\partial\bm{E}}{\partial t}
	\end{equation}
	Kramer vector:
	\begin{equation}\bm{F}=\bm{E}+i\bm{B}\end{equation}
	then we have:
	\begin{equation}
		\bm{\nabla}\times\bm{F}=\bm{\nabla}\times\bm{E}+i\bm{\nabla}\times\bm{B}
		=-\frac{1}{c}\frac{\partial\bm{B}}{\partial t}+\frac{i}{c}\frac{\partial\bm{E}}{\partial t}
		=\frac{i}{c}\frac{\partial\bm{F}}{\partial t}
	\end{equation}
	from the definition:
	\begin{equation}
		(\bm{\nabla}\times\bm{F})_\alpha=\epsilon_{\alpha\beta\gamma}\partial_\beta F_\gamma
	\end{equation}
	and:
	\begin{equation}p_\beta=\frac{\hbar}{i}\partial_\beta\end{equation}
	and:
	\begin{equation}\bm{S}=(S^1,S^2,S^3)\end{equation}
	where:
	\begin{equation}S^\beta_{\alpha\gamma}=i\epsilon_{\alpha\beta\gamma}\end{equation}
	Levi-Civita Symbol:
	\begin{equation}
		\epsilon_{\alpha\beta\gamma}=
		\begin{cases}
		+1 & \text{if } (i,j,k) \text{ is } (1,2,3), (2,3,1), \text{ or } (3,1,2), \\
       		-1 & \text{if } (i,j,k) \text{ is } (3,2,1), (1,3,2), \text{ or } (2,1,3), \\
    		0 & \text{if } i = j, \text{ or } j = k, \text{ or } k = i
    		\end{cases}
	\end{equation}
	\begin{equation}
		S^1= i\begin{bmatrix}0&0&0\\0&0&-1\\0&1&0\end{bmatrix},
		S^2= i\begin{bmatrix}0&0&1\\0&0&0\\-1&0&0\end{bmatrix},
		S^3= i\begin{bmatrix}0&-1&0\\1&0&0\\0&0&0\end{bmatrix}
	\end{equation}
	\begin{equation}
		\bm{S}\cdot\bm{S}= 2\begin{bmatrix}1&0&0\\0&1&0\\0&0&1\end{bmatrix},
		\bm{S}\times\bm{S}= i\bm{S}
	\end{equation}
	So:
	\begin{equation}
		(\hat{\bm{p}}\cdot\bm{S})F_\alpha=p_\beta S^\beta_{\alpha\gamma}F_\gamma
		=p_\beta(i\epsilon_{\alpha\beta\gamma})F_\gamma
		=\frac{\hbar}{i}(i\epsilon_{\alpha\beta\gamma})\partial_\beta F_\gamma
		=\hbar(\bm{\nabla}\times\bm{F})_\alpha
		=\hbar\frac{i}{c}\frac{\partial F_\alpha}{\partial t}
	\end{equation}	
	Quantum equation for photon:
	\begin{equation}
		c(\hat{\bm{p}}\cdot\bm{S})F=i\hbar\frac{\partial F}{\partial t}
	\end{equation}
\end{enumerate}
%----------------------------------------------------
\subsection{Dirac Picture}
2013-12-07
\subsubsection{Path integral}
\begin{enumerate}
\item Propagator as kernel of time-evolution of wave function \par
	Time operator $\hat{T}(t, t_0)$ is a unitary operator, so it has the equation:
	\begin{equation}
		i\hbar\frac{d\hat{T}(t, t_0)}{dt}=\hat{H}(t)\hat{T}(t,t_0)
	\end{equation}
	and:
	\begin{equation}
		\hat{T}(t, t')\hat{T}(t', t_0)=\hat{T}(t, t_0)
	\end{equation}
	It connects two states in Schr\"{o}dinger picture:
	\begin{equation}
		|A,t\rangle_S=\hat{T}(t, t')|A,t'\rangle_S
	\end{equation}
	Now assume $\hat{H}(t)=\hat{H}$, then we have:
	\begin{equation}
		\hat{T}(t, t')=e^{-i\hat{H}(t-t')/\hbar}
	\end{equation}
	\begin{equation}
		\psi(x,t)=\langle x|A, t\rangle_S=\langle x|\hat{T}(t, t')|A,t'\rangle_S
		=\int dx'\langle x|\hat{T}(t, t')|x'\rangle \langle|A,t'\rangle_S
		=\int dx'K(xt,x't')\psi(x',t')
	\end{equation}
	where:
	\begin{equation}
		K(xt,x't') = \langle x|\hat{T}(t, t')|x'\rangle =\langle x| e^{-i\hat{H}(t-t')/\hbar} |x'\rangle
	\end{equation}
\item The moving basis of coordinate representation

\item The infinitesimal propagator
\item The Feynman path integral
\item Path integral of free particle
\end{enumerate}
\subsubsection{Green's function}
\begin{enumerate}
\item Propagator and time-dependent Green's function
	\begin{equation}
		iG(xt,x't')=\langle x|\hat{T}(t,t')|x' \rangle
	\end{equation}
\item Differential equation of time-dependent Green's function
	\begin{equation}
		i\hbar \frac{\partial}{\partial t}G(xt,x't')=\hat{H}_x G(xt,x't')
	\end{equation}
\item Time-retarded and time-advanced Green's function \par
	Time-retarded Green Function:
	\begin{equation}
		G^{(+)}(xt,x't')=
		\begin{cases}
		G(xt,x't') & \text{if } t > t', \\
        		0 & \text{if } t < t'
    		\end{cases}
	\end{equation}
	Time-advanced Green Function:
	\begin{equation}
		G^{(-)}(xt,x't')=
		\begin{cases}
		0 & \text{if } t > t', \\
        		-G(xt,x't') & \text{if } t < t'
    		\end{cases}
	\end{equation}	
\item Heaviside step function\par
	Heaviside step function:
	\begin{equation}
		\theta (x)=
		\begin{cases}
		1 & \text{if } x > 0, \\
        		0 & \text{if } x < 0
    		\end{cases}
	\end{equation}	
	So we can express the Green function as:
	\begin{equation}\begin{split}
		G^{(+)}(xt,x't')&=\theta (t-t')G(xt,x't') \\
		G^{(-)}(xt,x't')&=-\theta (t'-t)G(xt,x't')
	\end{split}\end{equation}	
	An important expression of Heaviside step function:
	\begin{equation}
		\theta (x)=-\frac{1}{2\pi i}\int_{-\infty}^{+\infty}\frac{e^{-i\epsilon x}}{\epsilon+i \eta}d\epsilon
	\end{equation}	
	where $\eta=0^{+}$. \par
	From Sokhatsky-Weierstrass theorem:
	\begin{equation}
		\int^{g_1}_{-g_2}\frac{f(x)}{x-i \eta}dx=P\int^{g_1}_{-g_2}\frac{f(x)}{x}dx+i\pi f(0)
	\end{equation}	
	we have this important equation:
	\begin{equation}
		\frac{1}{x-i \eta}dx=\frac{P}{x}+i\pi \sigma(x)
	\end{equation}		
\item Differential equation of time-retarded and time-advanced GF
	\begin{equation}
		(i\hbar \frac{\partial}{\partial t}-\hat{H}_x)G^{(\pm)}(xt,x't')=\hbar \theta(t-t')\theta(x-x')
	\end{equation}		
\item Fourier transformation of time-retarded and time-advanced GF \par
	The Fourier transformation of delta function:
	\begin{equation}
		\delta(x)=\int_{-\infty}^{+\infty}\frac{dq}{2\pi}e^{iqx}
	\end{equation}	
	And for green function:
	\begin{equation}
		G^{(\pm)}(xt,x't')=\int_{-\infty}^{+\infty}\frac{d\omega}{2\pi}e^{-i\omega(t-t')}G^{(\pm)}(x,x';\omega)
	\end{equation}	\par
	So we have teh differential equation of time-retarded and time-advanced GF:
	\begin{equation}
		(\hbar \omega-\hat{H}_x)G^{(\pm)}(x,x',\omega)=\hbar \delta(x-x')
	\end{equation}	\par	
\item Green's functions as coordinate representation of Green's operators\par
	\begin{equation}
		iG^{(+)}(x't',x''t'')=\langle x'|i\hat{G}^{(+)}(t',t'')|x''\rangle
	\end{equation}	\par
	where:
	\begin{equation}
		i\hat{G}^{(+)}(t',t'')=\theta(t'-t'')\hat{T}(t',t'')
	\end{equation}	\par
	the Fourier transformation of it:
	\begin{equation}
		i\hat{G}^{(+)}(t',t'')=\int_{-\infty}^{+\infty}\frac{d\omega}{2\pi}e^{-i\omega(t'-t'')}i\hat{G}^{(+)}(\omega)
	\end{equation}	\par
	So we have:
	\begin{equation}
		G^{(\pm)}(x,x';\omega)=\langle x|\hat{G}^{(\pm)}(\omega)|x'\rangle
	\end{equation}	\par
	where:
	\begin{equation}
		G^{(\pm)}(\omega)=\frac{\hbar}{\hbar\omega-\hat{H}\pm i\eta}
	\end{equation}	\par
\end{enumerate}
\subsubsection{General properties of Green's function}
\begin{enumerate}
\item Green's function as resolvant of the eigenenergies
\item Analytical behavior of stationary state retarded and advanced Green's functions
\item Dyson's equation and transition operator
\end{enumerate}
\subsubsection{Green's function method for bound state and scattering state problems}
\begin{enumerate}
\item Zeroth order Green's functions
\item Finding bound state in vanishing potential well by GF method
\item Finding elastic scattering states by GF method
\end{enumerate}
\subsubsection{Lippmann-Schwinger equation and elastic potential scattering}
\begin{enumerate}
\item Lippmann-Schwinger equation for elastic potential scattering
\item The scattering amplitudes of short-range central potential field
\item Transition operator and Born approximation
\item Orthogonality relations
\end{enumerate}
\subsubsection{Dirac picture}
\begin{enumerate}
\item The Dirac(interaction)picture
\item Dyson operator and Tomonaga-Schwinger equation
\item Dyson series
\item The adiabatic ansatz of Born and Fock
\item The S-matrix of Wheeler
\item Relation between S-matrix and T-matrix
\item Fermi Golden rule first derived by Dirac
\end{enumerate}
%----------------------------------------------------
\subsection{Dirac second quantization method}
\subsubsection{Identical particle system}
\begin{enumerate}
\item Bosons: photon as the first one found by Planck, Einstein, Bose and Dirac
	
\item Bosons: the material particles
\item Fermions
\item Anyon and Laughlin ansatz
\item Spin-statistics connection
\end{enumerate}
\subsubsection{Quantum Statistics}
\begin{enumerate}
\item The probability density of statistical mechanics
\item The Gibbs ensemble of statistical mechanics
\item Liouville's theorem and equation
\item The mixed ensemble and density matrix of quantum statistics
\item Deriving Bose-Einstein statistics and Fermi-dirac statistics
\item Direct derivation of QM statistics
\item Path integral and partition function
\end{enumerate}
\subsubsection{Phonon and electron-phonon interaction}
\begin{enumerate}
\item The Born-Oppenheimer approximation and the harmonic approximation
\item Phonon as the quantized lattice vibrational mode
\item The electron-phonon interaction
\item The effective electron-electron interaction due to electron-phonon coupling
\end{enumerate}
\subsubsection{BCS theory of superconductivity}
\begin{enumerate}
\item The effective attraction
\item The Fermi surface
\item The BCS model of  superconductor
\item The mean-field approximation
\item The equation of motion and Nambu's discovery
\item The Bogoliubov-Valatin transformation
\item The ground state of BCS superconductor
\item The energy gap of BCS superconductor
\end{enumerate}
\subsubsection{Bogoliubov theory of superfluidity}
\begin{enumerate}
\item The interacting-boson model of superfluid of helium-4
\item The u-v transformation
\item The ground state
\end{enumerate}                     
%=======================================
\section{Quantum Field theory}

%=======================================
\section{General Relativity}

%=======================================
\section{Addition-A Units}
	Maxwell equation in Gauss convention:
	\begin{equation}
		\bm{\nabla}\cdot\bm{E}=4\pi \rho,\bm{\nabla}\cdot\bm{B}=0,	
		\bm{\nabla}\times\bm{E}=-\frac{1}{c}\frac{\partial\bm{B}}{\partial t},
		\bm{\nabla}\times\bm{B}=\frac{1}{c}\frac{\partial\bm{E}}{\partial t}+\frac{4\pi}{c} \bm{j}
	\end{equation} \par
	Maxwell equation in SI convention:
	\begin{equation}
		\bm{\nabla}\cdot\bm{E}=\rho/\epsilon_0,\bm{\nabla}\cdot\bm{B}=0,	
		\bm{\nabla}\times\bm{E}=-\frac{\partial\bm{B}}{\partial t},
		\bm{\nabla}\times\bm{B}=\frac{1}{c^2}\frac{\partial\bm{E}}{\partial t}+\mu_0 \bm{j}
	\end{equation}
%=======================================
\renewcommand\refname{Reference}
\bibliographystyle{plain}
\bibliography{DL} 
\clearpage
\end{document}