%\documentclass[12pt,draft]{article}
\documentclass[12pt]{article}
\usepackage{CJK}
\usepackage{mathrsfs}
\usepackage{amsmath,amsthm,amsfonts,amssymb}
\usepackage{geometry}
\usepackage{fancyhdr}
\usepackage{indentfirst}
\usepackage{float}
\usepackage[dvips]{graphicx}
\usepackage{subfigure}
\usepackage[font=small]{caption}
\usepackage{threeparttable}
\usepackage{cases}
\usepackage{multicol}
\usepackage{url}
\usepackage{amsmath}
\usepackage{bm}
\usepackage{xcolor}
\usepackage{overpic}
\usepackage{natbib}
\usepackage[utf8]{inputenc}
\usepackage[american]{babel}
\usepackage{natbib}
\usepackage{graphicx}
\numberwithin{equation}{section}
\geometry{left=1.5cm,right=1.5cm,top=1.5cm,bottom=1.5cm}
\setlength{\parskip}{0.3\baselineskip}
\setlength{\headheight}{15pt}
\begin{document}\small
 % \renewcommand\figurename{Fig.}
  %\renewcommand\arraystretch{1.0}
    \title{Math et Physics Notes}
    \author{Yan JIN}
    \pagestyle{fancy}\fancyhf{}
    \lhead{}\rhead{JIN Yan}
    \lfoot{\textit{}}\cfoot{}\rfoot{\thepage}
    \renewcommand{\headrulewidth}{1.pt}
    \renewcommand{\footrulewidth}{1.pt}
  \maketitle
  \tableofcontents
%=======================================
\section{Principles of Quantum Mechanics}
\subsection{Dirac's four axioms of Quantum Mechanics}
\subsubsection{Axiom I: Principle of superposition}
\begin{enumerate}
\item Definition of quantum states and the general principle of superposition
\item Mathematical formulation of the principle
\item Dirac's notation for vectors: the ket
\item Dirac's introduction of inner product function and bra vectors
\item The dual relationship between ket and bra
\end{enumerate}
\subsubsection{Axiom II: Principle of observables}
\begin{enumerate}
\item Linear operators (q-numbers)
\item Operator operating on the bra vectors
\item Conjugate relations

If $\hat{H}^\dag=\hat{H}$, then we call $\hat{H}$ as \textbf{self-adjoint operator} or \textbf{Hermitian operator}.
\item Eigenvalues, eigenvectors and eigenspace
\item The eigenvalue problem of Hermitian operators

For \textbf{Hermitian operator} $\hat{H}$:
\begin{enumerate}
	\item $\hat{H}$'s eigenvalues are real;
	\item $\hat{H}$'s eigenvectors for different eigenvalues are orthogonal.
\end{enumerate}
\item Axioms of observables in quantum mechanics and explanation of the Stern-Gerlach experiment

The connections between Quantum Theory and Experiments:
\begin{enumerate}
	\item Hermitian operators: \textbf{Observables};  
	\item Eigenvectors: The states of the system after measurements;
	\item Corresponding eigenvalues: The measurements' values from experiments.
\end{enumerate}
\end{enumerate}
\subsubsection{Axiom III: Quantization conditions}
\begin{enumerate}
\item Sequential Stern-Gerlach experiment again
\item Commutability and compatibility
\item Uncertainty relation between incompatible observables
\item Axiom of quantization conditions: Dirac canonical quantization
\item Heisenberg uncertainty relation between x and p
\item Uncertainty between x and p in the eigenenergy states of Planck oscillator
\end{enumerate}
\subsubsection{Axiom IV: Equation of motion}
\begin{enumerate}
\item The Heisenberg equation of motion
\item The Schr\"{o}dinger equation of motion
\end{enumerate}
%----------------------------------------------------
\subsection{Dirac's three rules of manipulations in Quantum Mechanics: 
                       Representations, Transformations and Pictures}
\subsubsection{Representations of discrete eigenvalue spectra - matrix mechanics}
\begin{enumerate}
\item The basis of a linear vector space and the basis vectors
\item The eigenvectors of Hermitian operators as orthonormal basis of Hilbert space
\item The discrete eigenvalue spectra and the matrix representation or matrix mechanics
\item Matrix (energy or Heisenberg) representation of Planck oscillator
\item Matrix representation of spin one half and the Stern-Gerlach experiment again
\end{enumerate}
\subsubsection{Transformations}
\begin{enumerate}
\item Spin operators along arbitrary direction and rotation of the basis vectors
\item Transformation of representations of discrete eigenvalue spectra
\item Infinitesimal spatial displacement and momentum operator
\item Finite translation as unitary transformation
\item Properties of unitary operator, infinitesimal unitary transformation and Hermitian operator, symmetry and invariance
\end{enumerate}
\subsubsection{Representations of continuous eigenvalue spectra}
\begin{enumerate}
\item Normalization of basis eigenvectors of the abstract linear vector space
\item Dirac delta function
\end{enumerate}
\subsubsection{The coordinate representation and the wave function}
\begin{enumerate}
\item The basis and the wave function
\item The coordinate representation of momentum operator
\item The "matrix elements" of the canonical quantization condition in the coordinate representation
\item The eigenfunction of momentum operator
\end{enumerate}
\subsubsection{Derivation of stationary state Schr\"{o}dinger's wave equation}
\begin{enumerate}
\item The non-relativistic Hamiltonian
\item The eigenequation of the Hamiltonian
\item The coordinate representation of the eigenequation of the Hamiltonian and
    the stationary state Schr\"{o}dinger's wave equation
\end{enumerate}
\subsubsection{Solution of eigenfunctions of Planck oscillator}
\begin{enumerate}
\item The wave function of the ground state   
\item The wave function of all eigenstates      
\end{enumerate}
\subsubsection{Representations of mixing discrete and continuous eigenvalue spectra}
\subsubsection{Complete set of dynamical quantities and simultaneous measurement of compatible observables}
\subsubsection{Pictures and Axiom IV of quantum mechanics-quantum dynamics}
\begin{enumerate}
\item The time evolution of quantum states
\item The Schr\"{o}dinger equation of motion of quantum states
\item The Schr\"{o}dinger picture
\item Derivation of the time-dependent Schr\"{o}dinger wave equation
\item The stationary state 
\item The Heisenberg picture
\item The Heisenberg equation of motion of Planck oscillator
\item The Ehrenfest theorem
\item Conservation laws in quantum mechanics
\end{enumerate}
\subsubsection{Coherent states of Planck oscillator}
\begin{enumerate}
\item The coherent state
\item The fourth Dirac ladder operators
\item Expansion in Heisenberg representation
\item Over-completeness
\item Non-orthogonality
\item The least uncertainty states in QM
\item The time-dependent coherent state
\item The coherent state Gaussian wave packet
\end{enumerate}
%----------------------------------------------------
\subsection{Dirac's theory of electron}
\subsubsection{Dirac equation}
\begin{enumerate}
\item The special theory of relativity
\item The four vectors and mass-energy equivalence \\
	\begin{equation}x_1=x, x_2=y, x_3=z, x_4=ict;\end{equation}
	\begin{equation}x_\mu x_\mu = \sum_{\mu=1}^{4} x_\mu x_\mu = x^2-c^2 t^2; \end{equation}
	\begin{equation}p_1=p_x, p_2=p_y, p_3=p_z, p_4=i\frac{E}{c};\end{equation}
	\begin{equation}p_\mu p_\mu = \sum_{\mu=1}^{4} p_\mu p_\mu = p^2-\frac{E^2}{c^2};\end{equation}
	Lorenz transformation requires that $x_\mu x_\mu$ and $p_\mu p_\mu$ etc. remain constant.\\
	Now, choose a special coordinate with $p'=0$ and $E'=E_0$, then 
	$p_\mu p_\mu=p'_\mu p'_\mu=0-\frac{E_0^2}{c^2}$, and define $m_0=\frac{E_0^2}{c^2}$, 
	then we have the mass-energy equivalence:
	\begin{equation}E^2=c^2p^2+m_0^2c^4\end{equation}
	From the definition: $p=\frac{m_0v}{\sqrt{1-\frac{v^2}{c^2}}}$, we get:
	\begin{equation}E=mc^2=\frac{m_0c^2}{\sqrt{1-\frac{v^2}{c^2}}}\end{equation}
\item The Klein-Gordon equation \\
	Replace mass-energy equivalence equation with:
	\begin{equation}E \rightarrow i\hbar \frac{\partial}{\partial{t}}, 
	\hat{\mathbf{p}} \rightarrow \frac{\hbar}{i} \hat{\mathbf{\nabla}}\end{equation}
	we get the Klein-Gordon equation:
	\begin{equation}-\hbar^2\frac{\partial^2{\psi(\mathbf{x},t)}}{\partial{t}^2}=
	(-\hbar^2 c^2\hat{\mathbf{\nabla}}^2+m^2_{0}c^4)\psi(\mathbf{x},t)\end{equation}
	use nature unit: $\hbar=c=1$, and introduce $\partial_{\mu}=\frac{\partial}{\partial{x}_{\mu}}$, 
	we have a more elegant form of this equation:
	\begin{equation}(\partial_{\mu}\partial_{\mu}-m^2_0)\psi(\mathbf{x},t)=0\end{equation}
	Solve this equation with: 
	$\psi(\mathbf{x},t)=(2\pi\hbar)^{-\frac{3}{2}}e^{\frac{i}{\hbar} p_\mu x_\mu}=
	(2\pi\hbar)^{-\frac{3}{2}}e^{\frac{i}{\hbar} (\mathbf{p}\cdot\mathbf{x}-Et)}$, 
	we have:$E^2=c^2p^2+m_0^2c^4$, that means the eigenvalues have two values:
	\begin{equation}E=\pm\sqrt{c^2p^2+m_0^2c^4}\end{equation}
\item Derivation of Dirac equation based on the four axioms of QM
	\begin{equation}\hat{H}=\sqrt{c^2\hat{p}^2+m_0^2c^4}
	=c\hat{\mathbf{\alpha}}\cdot\hat{\mathbf{p}}+\hat{\beta} m_0c^2\end{equation}
	that means:
	\begin{equation} c^2\hat{p}^2+m_0^2c^4
	=(c\hat{\mathbf{\alpha}}\cdot\hat{\mathbf{p}}+\hat{\beta} m_0c^2)^2\end{equation}
	Then we have:
	\begin{equation}\begin{split}
	\hat{\alpha_1^2}=\hat{\alpha_2^2}=\hat{\alpha_3^2}=\hat{\beta^2}=1 \\
	\hat{\mathbf{\alpha}}\hat{\beta}+\hat{\beta}\hat{\mathbf{\alpha}}=0 \\
	\hat{\alpha_1}\hat{\alpha_2}+\hat{\alpha_2}\hat{\alpha_1}=0 \\
	\hat{\alpha_2}\hat{\alpha_3}+\hat{\alpha_3}\hat{\alpha_2}=0 \\
	\hat{\alpha_3}\hat{\alpha_1}+\hat{\alpha_1}\hat{\alpha_3}=0 \\
	\end{split}\end{equation}
	Replace this into the Schr\"{o}dinger equation of motion, then we have dirac equation of motion:
	\begin{equation}i\hbar\frac{d|A,t>}{dt}=
	(c\hat{\mathbf{\alpha}}\cdot\hat{\mathbf{p}}+\hat{\beta} m_0c^2)|A,t>\end{equation}
	For $\hat{\mathbf{\alpha}}$ and $\hat{\beta}$, we choose a 4-dimension vector space, and the base vectors are:
	\[|1>=\begin{bmatrix}1\\ 0\\ 0\\ 0\\\end{bmatrix}, |2>=\begin{bmatrix}0\\ 1\\ 0\\ 0\\\end{bmatrix},
	|3>=\begin{bmatrix}0\\ 0\\ 1\\ 0\\\end{bmatrix}, |4>=\begin{bmatrix}0\\ 0\\ 0\\ 1\\\end{bmatrix}\]
	To calculate the dirac equation of motion, we choose the base vectors as:
	\begin{equation}|x,y,z;n>=|n>|x,y,z>\end{equation}
	where,
	\[|x,y,z;1>=\begin{bmatrix}|x,y,z>\\ 0\\ 0\\ 0\\ \end{bmatrix},
	|x,y,z;2>=\begin{bmatrix}0\\ |x,y,z>\\ 0\\ 0\\ \end{bmatrix},
	|x,y,z;3>=\begin{bmatrix}0\\ 0\\ |x,y,z>\\ 0\\ \end{bmatrix},
	|x,y,z;4>=\begin{bmatrix}0\\ 0\\ 0\\ |x,y,z>\\ \end{bmatrix}\]
	These base vectors are orthogonal and complete, so for each state vector of Dirac's electron $|A,t>$:
	\[|A,t>=\sum_{n=1}^4\iiint dxdydx |x,y,z;n><x,y,z;n|A,t>=\sum_{n=1}^4\iiint dxdydx\psi_n(x,y,z,t) |x,y,z;n>\]
	where,
	\[\psi_n(x,y,z,t)=<x,y,z;n|A,t>\]
	Then the dirac equation of motion becomes :
	\begin{equation}i\hbar\frac{\partial{\psi}}{dt}=
	(c\hat{\mathbf{\alpha}}\cdot\frac{\hbar}{i}\hat{\mathbf{\nabla}}+\hat{\beta} m_0c^2)\psi\end{equation}
	where,
	\[\psi=\psi(x,y,z,t)=\begin{bmatrix}\psi_1(x,y,z,t) \\ \psi_2(x,y,z,t) \\ \psi_3(x,y,z,t) \\ \psi_4(x,y,z,t) \\\end{bmatrix}
	=\begin{bmatrix}\psi_1 \\ \psi_2 \\ \psi_3 \\ \psi_4 \end{bmatrix}\]
\item Playing with the Pauli matrices \\
	Let $\hat{\alpha}_4=\hat{\beta}$, we could rewrite the relations using anti-exchange notation:
	\begin{equation}\{\hat{\alpha}_\mu, \hat{\alpha}_\nu \}=
	\hat{\alpha}_\mu \hat{\alpha}_\nu + \hat{\alpha}_\nu \hat{\alpha}_\mu = 
	2 \delta_{\mu \nu}\end{equation}
	This is just like the relation between pauli matrix.\\
	Now we assume $m_0=0$, the static mass is zero:
	\begin{equation}\hat{H}=\sqrt{c^2\hat{p}^2+m_0^2c^4}
	=c\sqrt{\hat{p}^2}=c\sqrt{\hat{\mathbf{p}}\cdot\hat{\mathbf{p}}}
	=c\sqrt{(\hat{\mathbf{\sigma}}\cdot\hat{\mathbf{p}})^2}
	=c(\hat{\mathbf{\sigma}}\cdot\hat{\mathbf{p}})\end{equation}
	Replace it into the the Schr\"{o}dinger equation of motion:
	\begin{equation}i\hbar\frac{d|A,t>}{dt}
	=c(\hat{\mathbf{\sigma}}\cdot\hat{\mathbf{p}})|A,t>\end{equation}
	This is the equation for particles with spin 1/2 and zero static mass. If we constrain the $\hat{\mathbf{p}}=(\hat{p}_x,\hat{p}_y)$, then it's the 2010 Nobel Prize's work on graphene, the weyl equation. \\
	Pauli matrix:
	\begin{equation}
	\hat{\sigma}_x=\begin{bmatrix}0&1\\1&0\end{bmatrix},
	\hat{\sigma}_x=\begin{bmatrix}0&-i\\i&0\end{bmatrix},
	\hat{\sigma}_x=\begin{bmatrix}1&0\\0&-1\end{bmatrix}
	\end{equation}
	it has relation:
	\begin{equation}
	(\hat{\mathbf{\sigma}}\cdot\mathbf{a})(\hat{\mathbf{\sigma}}\cdot\mathbf{b})
	=\mathbf{a}\cdot\mathbf{b}+i\hat{\mathbf{\sigma}}\cdot(\mathbf{a}\times\mathbf{b})
	\end{equation}
\item Dirac equation in Dirac representation
	\begin{equation}
	\hat{\mathbf{\alpha}}=\begin{bmatrix}0&\hat{\mathbf{\sigma}}\\\hat{\mathbf{\sigma}}&0\end{bmatrix},
	\hat{\mathbf{\beta}}=\begin{bmatrix}\hat{\mathbf{1}}&0\\0&-\hat{\mathbf{1}}\end{bmatrix},
	\end{equation}
	so we have the Dirac equation in Dirac representation:
	\begin{equation}
	i\hbar\frac{\partial \psi}{\partial t}=
	[c\begin{bmatrix}0&\hat{\mathbf{\sigma}}\\\hat{\mathbf{\sigma}}&0\end{bmatrix}\cdot(\frac{\hbar}{i}\hat{\mathbf{\nabla}})+\begin{bmatrix}\hat{\mathbf{1}}&0\\0&-\hat{\mathbf{1}}\end{bmatrix}m_0c^2]\psi
	\end{equation}
\item Dirac equation in Weyl representation
	\begin{equation}
	\hat{\mathbf{\alpha}}=\begin{bmatrix}\hat{\mathbf{\sigma}}&0\\0&-\hat{\mathbf{\sigma}}\end{bmatrix},
	\hat{\mathbf{\beta}}=\begin{bmatrix}0&\hat{\mathbf{1}}\\\hat{\mathbf{1}}&0\end{bmatrix},
	\end{equation}
	so we have the Dirac equation in Weyl representation:
	\begin{equation}
	i\hbar\frac{\partial \psi}{\partial t}=
	[c\begin{bmatrix}\hat{\mathbf{\sigma}}&0\\0&-\hat{\mathbf{\sigma}}\end{bmatrix}\cdot(\frac{\hbar}{i}\hat{\mathbf{\nabla}})+\begin{bmatrix}0&\hat{\mathbf{1}}\\\hat{\mathbf{1}}&0\end{bmatrix}m_0c^2]\psi
	\end{equation}
	Mr. Nambu found the similarity between the Weyl representation and the BCS superconductivity theory, and do his Nobel Prize work.
\end{enumerate}
\subsubsection{The motion of a free electron}
\begin{enumerate}
\item The velocity of Dirac electron and zitterbewegung
\item The current conservation of a Dirac electron
\item Conservation of angular momentum of a Dirac electron and spin
\item Complete set of dynamical quantities of Dirac electron and helicity
\item The plane wave solution of a free Dirac electron
\end{enumerate}
\subsubsection{Dirac sea, Dirac hole and Modern Concept of Particles}
\begin{enumerate}
\item The concept of Dirac sea based on the Pauli exclusion principle
\item The assumption about observables   
\item The concept of particles in QM
\item Dirac's hole theory, concept of anti-particle and prediction of positron
\item The electron-positron pair production
\end{enumerate}
\subsubsection{Quantum field theoretical formulation of Dirac hole theory}
\begin{enumerate}
\item The energy of many Dirac electrons as eigenenergy of a new Hamiltonian
\item The Fock representation
\item The quantum field theoretical Hamiltonian of electrons and positrons
\item The second quantization method
\end{enumerate}
\subsubsection{The non-relativistic limit of Dirac equation}
\begin{enumerate}
\item A Dirac electron in electromagnetic field
\item Eliminating the rest mass of electron
\item The Pauli equation
\item The magnetic moment of Dirac electron
\item The spin-orbit coupling
\end{enumerate}
\subsubsection{Covariance of Dirac equation under symmetrical transformations}
\begin{enumerate}
\item Covariance of Dirac equation under Lorentzian and rotational transformations
\item Dirac matrix, Dirac algebra and Dirac spinor
\item Lorentzian transformations
\item Rotational transformation around z axis
\item Properties of the coordinate transformation matrices
\item Coordinate transformation of Dirac equation
\item Direct transformation of Dirac spinor and Dirac matrices
\item Combined rotational transformations of Dirac spinor
\end{enumerate}
\subsubsection{Neutrino, Dirac Fermion, Photon}
\begin{enumerate}
\item Neutrino
\item Dirac Fermion
\item Spin of photon
\end{enumerate}
%----------------------------------------------------
\subsection{Dirac Picture}
\subsubsection{Path integral}
\begin{enumerate}
\item Propagator as kernel of time-evolution of wave function
\item The moving basis of coordinate representation
\item The infinitesimal propagator
\item The Feynman path integral
\item Path integral of free particle
\end{enumerate}
\subsubsection{Green's function}
\begin{enumerate}
\item Propagator and time-dependent Green?s function
\item Differential equation of time-dependent Green?s function
\item Time-retarded and time-advanced Green?s function
\item Heaviside step function
\item Differential equation of time-retarded and time-advanced GF
\item Fourier transformation of time-retarded and time-advanced GF
\item Green?s functions as coordinate representation of Green?s operators
\end{enumerate}
\subsubsection{General properties of Green's function}
\begin{enumerate}
\item Green?s function as resolvant of the eigenenergies
\item Analytical behavior of stationary state retarded and advanced Green?s functions
\item Dyson?s equation and transition operator
\end{enumerate}
\subsubsection{Green's function method for bound state and scattering state problems}
\begin{enumerate}
\item Zeroth order Green?s functions
\item Finding bound state in vanishing potential well by GF method
\item Finding elastic scattering states by GF method
\end{enumerate}
\subsubsection{Lippmann-Schwinger equation and elastic potential scattering}
\begin{enumerate}
\item Lippmann-Schwinger equation for elastic potential scattering
\item The scattering amplitudes of short-range central potential field
\item Transition operator and Born approximation
\item Orthogonality relations
\end{enumerate}
\subsubsection{Dirac picture}
\begin{enumerate}
\item The Dirac(interaction)picture
\item Dyson operator and Tomonaga-Schwinger equation
\item Dyson series
\item The adiabatic ansatz of Born and Fock
\item The S-matrix of Wheeler
\item Relation between S-matrix and T-matrix
\item Fermi Golden rule first derived by Dirac
\end{enumerate}
%----------------------------------------------------
\subsection{Dirac second quantization method}
\subsubsection{Identical particle system}
\begin{enumerate}
\item Bosons: photon as the first one
\item Bosons: the material particles
\item Fermions
\item Anyon and Laughlin ansatz
\item Spin-statistics connection
\end{enumerate}
\subsubsection{Quantum Statistics}
\begin{enumerate}
\item The probability density of statistical mechanics
\item The Gibbs ensemble of statistical mechanics
\item Liouville's theorem and equation
\item The mixed ensemble and density matrix of quantum statistics
\item Deriving Bose-Einstein statistics and Fermi-dirac statistics
\item Direct derivation of QM statistics
\item Path integral and partition function
\end{enumerate}
\subsubsection{Phonon and electron-phonon interaction}
\begin{enumerate}
\item The Born?Oppenheimer approximation and the harmonic approximation
\item Phonon as the quantized lattice vibrational mode
\item The electron-phonon interaction
\item The effective electron-electron interaction due to electron-phonon coupling
\end{enumerate}
\subsubsection{BCS theory of superconductivity}
\begin{enumerate}
\item The effective attraction
\item The Fermi surface
\item The BCS model of  superconductor
\item The mean-field approximation
\item The equation of motion and Nambu's discovery
\item The Bogoliubov-Valatin transformation
\item The ground state of BCS superconductor
\item The energy gap of BCS superconductor
\end{enumerate}
\subsubsection{Bogoliubov theory of superfluidity}
\begin{enumerate}
\item The interacting-boson model of superfluid of helium-4
\item The u-v transformation
\item The ground state
\end{enumerate}                     
%=======================================
\section{Quantum Field theory}

%=======================================
\section{General Relativity}

%=======================================
\section{Geometry}
\subsection{Set X}
	\textbf{Set}: distinction between elements.
\subsection{Topology(X,T)}
	\begin{enumerate}
	\item \textbf{Topology}: define what is a \textbf{open subset}, which separated from other subsets of set X;
	\item A set with this definition(distinction between subsets of set X) above is called \textbf{topological space};
	\item So we can define \textbf{continuous} on the mappings of different Topologies(open subsets).
	\item \textbf{Homeomorphism}: 
		\begin{enumerate}
		\item one-one-onto; 
		\item \textbf{continuous}($C^0$) forward and backward;
		\end{enumerate}
	\item Two Topologies \textbf{homeomorphic} to each other means, they are equal on the basis of 	\textbf{Topology};
	\end{enumerate}
\subsection{Manifold}
	\begin{enumerate}
	\item n-dimensional differentiable manifold M: 
		\begin{enumerate}
		\item exist a \textbf{open cover} of M, such that, for every \textbf{open subset} of the open cover, exist a homeomorphism to a open subset of $R^n$; 
		\item the mappings from any two different coordinate systems are \textbf{infinity differencial}($C^\infty$);
		\end{enumerate}
	\item \textbf{Diffeomorphism}: 
		\begin{enumerate}
		\item one-one-onto; 
		\item \textbf{infinity differential}($C^\infty$) forward and backward;
		\end{enumerate}
	\item Two Manifolds \textbf{diffeomorphic} to each other means, they are equal on the basis of "Manifold";
	\end{enumerate}
\subsection{Group}
	\begin{enumerate}
	\item \textbf{Group}: set G with a "product" $G\times G\rightarrow G$:associative, identity element, inverse element
	\item \textbf{Homomorphism}: keep the "product" relationship;
	\item \textbf{Isomorphism}: homomorphism + one-one-onto;
	\item Two Group \textbf{isomorphic} to each other means, they are equal on the basis of Group;
	\end{enumerate}
\subsection{Lie Group}
	\textbf{Lie group}: Group + Manifold + $C^\infty$ on "product" and "inverse" mappings;
\subsection{Algebra}
	\textbf{Algebra}: Vector space + product;
\subsection{Lie Algebra}
	\textbf{Lie algebra}: Vector space + Lie bracket(a special "product");	
\subsection{continue}
	\textbf{Scalar field on M}(function on M): $f:M \rightarrow R$; \par
	$\mathfrak{F}_M$: all the scalar fields on M; \par
	\textbf{Vector}: $\mathfrak{F}_M\rightarrow R$; \par
	$V_p$: all the vectors on M of point p; \par
	$X_\mu(f)=\frac{\partial f(x)}{\partial x^\mu}|_p = \frac{\partial F(x^1,...,x^n))}{\partial x^\mu}|_p, \forall f \in \mathfrak{F}_M$. \par
	${X_\mu}$ of point p are coordinate basis, $v^\mu=v(x^\mu)$ are coordinate components of v;\par
	\textbf{Curve}: a mapping of $I\rightarrow M, I\subseteq R$.\par
	\textbf{Tangent vector}: $T(f):=\frac{d(f\circ C))}{dt}|_{t_0}, \forall f \in \mathfrak{F}_M$.\par
	$X_\mu$ is a "tangent vector" of $x^\mu$ coordinate line on point p;\par
	All the elements of $V_p$ are also called \textbf{tangent vector}, v is called \textbf{tangent space} of point p;

\renewcommand\refname{Reference}
\bibliographystyle{plain}
\bibliography{DL}

  \clearpage
\end{document}